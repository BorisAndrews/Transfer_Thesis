\section{Kinetic Models: The General Theory}
    \BA{What is a ``kinetic'' model?}
    
    \begin{definition}[Kinetic model]
        Here, ``kinetic'' model refers to a model wherein a fluid is modelled via a particle density function of position and velocity (and time).
    \end{definition}
    
    This is typically written as $f(\bfx, \bfv; t)$. Obviously, with the high dimensionality, this can be computationally prohibitive, so we'd like to avoid such a model if possible.
    
    \BA{Boltzmann equation definition here.}
    
    \begin{definition}[Fluid model]
        Here, ``fluid'' model refers to a reduced kinetic model, where a collection of functions of position (and time) only are modeled.
    \end{definition}
    
    These typically define certain properties of the kinetic model density function, $f$, at each position, $\bfx$, to in some way capture the physics of the full kinetic model. The Navier–Stokes equations, for example, can be found from the assumption that collisions between particles dominate the behavior of each particle, such that at each position, $\bfx$, the kinetic model density function, $f|_{\bfx}$, converges to a high entropy state (often a scaled normal distribution: a Maxwellian) characterized by 3 conserved variables:
    \begin{itemize}
        \item  Mass (per unit volume, i.e. density)
        \item  Momentum (per unit volume)
        \item  Energy (per unit volume, i.e. temperature after conservation of mass of momentum)
    \end{itemize}
    When collisions between particles do \emph{not} dominate the particle behavior—as is the case in the very low-density, high-temperature \BA{(Does this increase or decrease the size of the Coulomb collision term?)} tokamak plasmas—the assumptions that give rise to these fluid approximations often break down, implying that, without modification, these models fail to capture so-called ``kinetic effects'' \BA{(Such as?)}.
    
    \begin{figure}[!h]
        \centering
        \begin{tikzpicture}[align = center, node distance = 4cm, auto]
            \node[1] (11) at (0, 0) {{\bf Fluid} Single-Phase Model (Navier–Stokes)};
            \node[2] (12) at (0.5, -1.75) {{\bf Kinetic} Single-Phase Model};
            \node[2] (21) at (5.5, 0) {{\bf Fluid} Multi-phase Model};
            \node[3] (22) at (6, -1.75) {{\bf Kinetic} Multi-phase model};
            \node[3] (31) at (11, 0) {{\bf Fluid} Plasma Model (MHD)};
            \node[4] (32) at (11.5, -1.75) {{\bf Kinetic} \\ Plasma Model};
    
            \path[arrow] (11) -- (12);
            \path[arrow] (11) -- (21);
            \path[arrow] (12) -- (22);
            \path[arrow] (21) -- (22);
            \path[arrow] (21) -- (31);
            \path[arrow] (22) -- (32);
            \path[arrow] (31) -- (32);
        \end{tikzpicture}
        \caption{\BA{Diagram of workflow for creating a kinetic plasma model to account for pressure anisotropy.}}
    \end{figure}
    
    
\subsection{Single-Phase Fluids}
    \BA{The resultant kinetic PDE. (Boltzmann equation.)}
                
    \BA{How we traditionally convert that to a ``fluid'' model.}
    
    \BA{Will use this simpler case as a reference study to develop the ideas for the more complicated tokamak plasma case.}
    
    \BA{The way I've got this section structured with reference to introducing the moments is a bit funny.}
    
    Consider first a single-phase fluid. Let a single particle of mass $m$ in this fluid have position $\bfx(t)$ and velocity $\bfv(t)$, and be subject to a force $\bfF(\bfx, \bfv; t)$. We seek a (system of) ODEs governing the behaviour of this particle:
    \begin{itemize}
        \item  The position evolves simply according to the ODE
        \begin{equation}\label{eqn:velocity ODE}
            d\bfx  =  \bfv dt
        \end{equation}
        \item  The velocity evolves according to the ODE, $d\bfv  =  \frac{1}{m}\bfF dt + \frac{1}{m}${\bf ``Collisional forces''}$dt$. \BA{(Note here with references about how tough it is to model collisional terms- \emph{got} to make \emph{some} assumptions, \emph{especially} if we want to do simulations!)} To model the collisional forces, consider the following 2 assumptions:
        \begin{itemize}
            \item  The molecular chaos hypothesis:
            \begin{definition}[Molecular chaos hypothesis]
                The ``molecular chaos hypothesis'' postulates that the velocities of colliding particles are uncorrelated, and independent of position. \BA{[Ref]} \BA{(Justification that this is valid here? Particle number sufficiently large?)}
            \end{definition}
            \item  Collisional forces on the particles are dominated by those from Coulomb collisions: \BA{(Is this the right way of phrasing this?)}
            \begin{definition}[Coulomb collision]
                A ``Coulomb collision'' is an elastic collision between two charged particles interacting through their own electric field. \BA{[Ref]} \BA{(Justification that this is valid here?)}
            \end{definition}
        \end{itemize}
        We can then model the collision forces through a Wiener process, with forces proportional to the:  \BA{(Give justification for this model. That might be tough...)}
        \begin{itemize}
            \item  Difference in momentum between that of the particle, $m\bfv$, and the average particle with which the particle in question is colliding, $m(\bfu + \nabla_{\bfx}\cdot\bftau)$ \BA{(I think this is right for the stress contribution? Maybe off by a scaling of $m$ or something...)}, where $\bftau(\bfx)$ denotes an internal stress from variation in momentum \BA{($\bftau$ should be symmetric, right? But why?)} \BA{(Should this be $\bftau$ or $\bfsigma$? I've written $\bftau$ so as not to clash in notation, but this could change.)} \BA{(Why?)}
            \begin{equation}\label{eqn:single-phase drag force}
                \mu m[(\bfu + \nabla_{\bfx}\cdot\bftau) - \bfv]
            \end{equation}
            \item  ``Derivative'' of some Wiener process \BA{(Obviously the ``derivative of a Wiener process'' is not a rigorously defined concept, so I may see if I can switch around the way this is structured...)}, representing the random forces from collisions at random angles from the molecular chaos hypothesis
            \begin{equation}\label{eqn:single-phase brownian motion}
                \sigma m\frac{d\bfW}{dt}
            \end{equation}
        \end{itemize}
        
        \BA{Here's where I think I should state that we have the relation $\frac{\mu}{\sigma^{2}}  =  \frac{k_{B}T}{m}$. Is this from conservation of energy? Would like to give a proper derivation of the scale of $\sigma$, and am sure I'll have to for the dimensional analysis.}
        
        By convention, mapping $\bfx, \bfv  \mapsto  \bfX, \bfV$ to represent the switch from an ODE to an SDE (and accordingly in (\ref{eqn:velocity ODE})) gives the SDE for the velocity evolution
        \begin{equation}
            d\bfV  =  \frac{1}{m}\bfF dt + \mu[(\bfu + \nabla_{\bfx}\cdot\bftau) - \bfV]dt + \sigma d\bfW
        \end{equation}
        This resembles a combination of an Ornstein–Uhlenbeck process and a forcing term.
    \end{itemize}
    Applying the Fokker-Planck equation, the particle density function, $f(\bfx, \bfv; t)$ evolves according to the PDE
    \begin{equation}\label{eqn:single-phase Boltzmann equation}
        \partial_{t}[f] + \underset{\textrm{Convection}}{\underbrace{\nabla_{\bfx}\cdot[f\bfv]}} + \underset{\textrm{Forces}}{\underbrace{\frac{1}{m}\nabla_{\bfv}\cdot[f\bfF]}}  =  - \underset{\textrm{Viscosity}}{\underbrace{\mu\nabla_{\bfv}\cdot\nabla_{\bfx}\cdot[\bftau]}} + \underset{\textrm{Collisions}}{\underbrace{\mu\nabla_{\bfv}\cdot[f(\bfv - \bfu)] + \frac{\sigma^{2}}{2}\Delta_{\bfv}[f]}}
    \end{equation}
    This is a form of the Boltzmann equation, where the approximation to the collisional term is explicitly derived from the above assumptions.
    \subsection{Multiphase Fluids}
    \BA{Similar analysis for a multiphase fluid, in preparation for handling the tozmahok plasmas.}

    Working towards the full plasma kinetic equation, consider a general multiphase fluid, where each phase is subject to a different external force. We index each phase, $i$, through the upper index $*^{(i)}$.

    The collisional forces (\ref{single-phase drag force}) and (\ref{single-phase brownian motion}) on phase $i$ will be modified to the form:  \BA{(This will definitely be switched around in the future when I switch around my notation for $\mu^{(i, j)}$, $\sigma^{(i)}$. In particular there should be some symmetry relations on $\mu^{(i, j)}$ from conservation of momentum between phases $i$, $j$- I'd like to scale $\mu^{(i, j)}$ such that $\mu^{(i, j)} = \mu^{(j, i)}$.)}
    \begin{align}
        \mu m[(\bfu + \nabla_{\bfx}\cdot\bftau) - \bfv]  &\mapsto  \sum_{j}\mu^{(i, j)}m^{(i)}\left[\left(\bfu^{(j)} + \nabla_{\bfx}\cdot\bftau^{(i, j)}\right) - \bfv\right]  \\
        \sigma m\frac{d\bfW}{dt}  &\mapsto  \sum_{i}\sigma^{(i)}m^{(j)}\frac{d\bfW^{(i)}}{dt}
    \end{align}
    \BA{We should similarly here have the relation $\frac{\mu^{(i)}}{{\sigma^{(i)}}^{2}} = \frac{k_{B}T^{(i)}}{m^{(i)}}$ I believe.}

    Accordingly, for each phase $i$ the single-phase Boltzmann equation (\ref{single-phase Boltzmann equation}) takes the multiphase form
    \begin{multline}\label{multiphase Boltzmann equation}
        \partial_{t}\left[f^{(i)}\right] + \underset{\textrm{Convection}}{\underbrace{\nabla_{\bfx}\cdot\left[f^{(i)}\bfv\right]}} + \underset{\textrm{Forces}}{\underbrace{\frac{1}{m^{(i)}}\nabla_{\bfv}\cdot\left[f^{(i)}\bfF^{(i)}\right]}}  \\  =  - \underset{\textrm{Viscosity}}{\underbrace{\sum_{j}\mu^{(i, j)}\nabla_{\bfv}\cdot\nabla_{\bfx}\cdot\left[\bftau^{(i, j)}\right]}} + \underset{\textrm{Collisions}}{\underbrace{\sum_{j}\mu^{(i, j)}\nabla_{\bfv}\cdot\left[f\left(\bfv - \bfu^{(j)}\right)\right] + \frac{(\sigma^{(i)})^{2}}{2}\Delta_{\bfv}\left[f^{(i)}\right]}}
    \end{multline}
    \subsection{Tokamak Plasmas}    
    \BA{The resultant kinetic PDE. (Boltzmann/Vlasov equations.)}
    
    \BA{Talk about gyrokinetic model:
    \begin{itemize}
        \item  The model's physical basis/mathematics. (Equations provide good insight into the origin of some behavioural effects, e.g. gyro-orbits/drifts.)
        \item  Why we don't use it on the general kinetic equation:
        \begin{itemize}
            \item  High mathematical (more terms in lower dimensions doesn't necessarily mean faster computation)/computational (really don't want to do a 5D simulation) complexity.
            \item  Errors from neglection of terms. (Non-physical behaviour over long times/resonances and adiabatic invariants can be lost.)
        \end{itemize}
        \item  We \emph{will} however use it for the PIC correction.
    \end{itemize}}