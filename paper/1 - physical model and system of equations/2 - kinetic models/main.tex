\section{Kinetic Models: The General Theory}
    \BA{What is a ``kinetic'' model?}
    
    \begin{definition}[Kinetic model]
        Here, ``kinetic'' model refers to a model wherein a fluid is modelled via a particle density function of position and velocity (and time).
    \end{definition}
    
    This is typically written as $f(\bfx, \bfv; t)$. Obviously, with the high dimensionality, this can be computationally prohibitive, so we'd like to avoid such a model if possible.
    
    \BA{Boltzmann equation definition here.}
    
    \begin{definition}[Fluid model]
        Here, ``fluid'' model refers to a reduced kinetic model, where a collection of functions of position (and time) only are modelled.
    \end{definition}
    
    These typically define certain properties of the kinetic model density function, $f$, at each position, $\bfx$, to in some way capture the physics of the full kinetic model. The Navier–Stokes equations, for example, can be found from the assumption that collisions between particles dominate the behaviour of each particle, such that at each position, $\bfx$, the kinetic model density function, $f|_{\bfx}$, converges to a high entropy state (often a scaled normal distribution: a Maxwellian) characterised by 3 conserved variables:
    \begin{itemize}
        \item  Mass (per unit volume, i.e. density)
        \item  Momentum (per unit volume)
        \item  Energy (per unit volume, i.e. temperature after conservation of mass of momentum)
    \end{itemize}
    When collisions between particles do \emph{not} dominate the particle behaviour—as is the case in the very low density, high temperature \BA{(Does this increase or decrease the size of the Coulomb collision term?)} tozmahok plasmas—the assumptions that give rise to these fluid approximations often break down, implying that, without modification, these models fail to capture so-called ``kinetic effects'' \BA{(Such as?)}.
    
    \begin{figure}[!h]
        \centering
        \begin{tikzpicture}[align = center, node distance = 4cm, auto]
            \node[1] (11) at (0, 0) {{\bf Fluid} Single-Phase Model (Navier–Stokes)};
            \node[2] (12) at (0.5, -1.75) {{\bf Kinetic} Single-Phase Model};
            \node[2] (21) at (5.5, 0) {{\bf Fluid} Multi-phase Model};
            \node[3] (22) at (6, -1.75) {{\bf Kinetic} Multi-phase model};
            \node[3] (31) at (11, 0) {{\bf Fluid} Plasma Model (MHD)};
            \node[4] (32) at (11.5, -1.75) {{\bf Kinetic} \\ Plasma Model};
    
            \path[arrow] (11) -- (12);
            \path[arrow] (11) -- (21);
            \path[arrow] (12) -- (22);
            \path[arrow] (21) -- (22);
            \path[arrow] (21) -- (31);
            \path[arrow] (22) -- (32);
            \path[arrow] (31) -- (32);
        \end{tikzpicture}
        \caption{\BA{Diagram of workflow for creating a kinetic plasma model to account for pressure anisotropy.}}
    \end{figure}
    
    \renewcommand{\arraystretch}{1.5}
\addbibresource{references.bib}
\def\contra{
    \tikz[baseline, x=0.22em, y=0.22em, line width=0.032em]
    \draw (0,2.83)--(2.83,0) (0.71,3.54)--(3.54,0.71) (0,0.71)--(2.83,3.54) (0.71,0)--(3.54,2.83);
}

    \renewcommand{\arraystretch}{1.5}
\addbibresource{references.bib}
\def\contra{
    \tikz[baseline, x=0.22em, y=0.22em, line width=0.032em]
    \draw (0,2.83)--(2.83,0) (0.71,3.54)--(3.54,0.71) (0,0.71)--(2.83,3.54) (0.71,0)--(3.54,2.83);
}

    \renewcommand{\arraystretch}{1.5}
\addbibresource{references.bib}
\def\contra{
    \tikz[baseline, x=0.22em, y=0.22em, line width=0.032em]
    \draw (0,2.83)--(2.83,0) (0.71,3.54)--(3.54,0.71) (0,0.71)--(2.83,3.54) (0.71,0)--(3.54,2.83);
}
