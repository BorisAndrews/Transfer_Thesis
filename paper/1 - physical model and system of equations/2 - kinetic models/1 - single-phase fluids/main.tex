
\subsection{Single-Phase Fluids}
    \BA{The resultant kinetic PDE. (Boltzmann equation.)}
                
    \BA{How we traditionally convert that to a ``fluid'' model.}
    
    \BA{Will use this simpler case as a reference study to develop the ideas for the more complicated tokamak plasma case.}
    
    \BA{The way I've got this section structured with reference to introducing the moments is a bit funny.}
    
    Consider first a single-phase fluid. Let a single particle of mass $m$ in this fluid have position $\bfx(t)$ and velocity $\bfv(t)$, and be subject to a force $\bfF(\bfx, \bfv; t)$. We seek a (system of) ODEs governing the behaviour of this particle:
    \begin{itemize}
        \item  The position evolves simply according to the ODE
        \begin{equation}\label{velocity ODE}
            d\bfx  =  \bfv dt
        \end{equation}
        \item  The velocity evolves according to the ODE, $d\bfv  =  \frac{1}{m}\bfF dt + \frac{1}{m}${\bf ``Collisional forces''}$dt$. \BA{(Note here with references about how tough it is to model collisional terms- \emph{got} to make \emph{some} assumptions, \emph{especially} if we want to do simulations!)} To model the collisional forces, consider the following 2 assumptions:
        \begin{itemize}
            \item  The molecular chaos hypothesis:
            \begin{definition}[Molecular chaos hypothesis]
                The ``molecular chaos hypothesis'' postulates that the velocities of colliding particles are uncorrelated, and independent of position. \BA{[Ref]} \BA{(Justification that this is valid here? Particle number sufficiently large?)}
            \end{definition}
            \item  Collisional forces on the particles are dominated by those from Coulomb collisions: \BA{(Is this the right way of phrasing this?)}
            \begin{definition}[Coulomb collision]
                A ``Coulomb collision'' is an elastic collision between two charged particles interacting through their own electric field. \BA{[Ref]} \BA{(Justification that this is valid here?)}
            \end{definition}
        \end{itemize}
        We can then model the collision forces through a Wiener process, with forces proportional to the:  \BA{(Give justification for this model. That might be tough...)}
        \begin{itemize}
            \item  Difference in momentum between that of the particle, $m\bfv$, and the average particle with which the particle in question is colliding, $m(\bfu + \nabla_{\bfx}\cdot\bftau)$ \BA{(I think this is right for the stress contribution? Maybe off by a scaling of $m$ or something...)}, where $\bftau(\bfx)$ denotes an internal stress from variation in momentum \BA{($\bftau$ should be symmetric, right? But why?)} \BA{(Should this be $\bftau$ or $\bfsigma$? I've written $\bftau$ so as not to clash in notation, but this could change.)} \BA{(Why?)}
            \begin{equation}\label{single-phase drag force}
                \mu m[(\bfu + \nabla_{\bfx}\cdot\bftau) - \bfv]
            \end{equation}
            \item  ``Derivative'' of some Wiener process \BA{(Obviously the ``derivative of a Wiener process'' is not a rigorously defined concept, so I may see if I can switch around the way this is structured...)}, representing the random forces from collisions at random angles from the molecular chaos hypothesis
            \begin{equation}\label{single-phase brownian motion}
                \sigma m\frac{d\bfW}{dt}
            \end{equation}
        \end{itemize}
        
        \BA{Here's where I think I should state that we have the relation $\frac{\mu}{\sigma^{2}}  =  \frac{k_{B}T}{m}$. Is this from conservation of energy? Would like to give a proper derivation of the scale of $\sigma$, and am sure I'll have to for the dimensional analysis.}
        
        By convention, mapping $\bfx, \bfv  \mapsto  \bfX, \bfV$ to represent the switch from an ODE to an SDE (and accordingly in (\ref{velocity ODE})) gives the SDE for the velocity evolution
        \begin{equation}
            d\bfV  =  \frac{1}{m}\bfF dt + \mu[(\bfu + \nabla_{\bfx}\cdot\bftau) - \bfV]dt + \sigma d\bfW
        \end{equation}
        This resembles a combination of an Ornstein–Uhlenbeck process and a forcing term.
    \end{itemize}
    Applying the Fokker-Planck equation, the particle density function, $f(\bfx, \bfv; t)$ evolves according to the PDE
    \begin{equation}\label{single-phase Boltzmann equation}
        \partial_{t}[f] + \underset{\textrm{Convection}}{\underbrace{\nabla_{\bfx}\cdot[f\bfv]}} + \underset{\textrm{Forces}}{\underbrace{\frac{1}{m}\nabla_{\bfv}\cdot[f\bfF]}}  =  - \underset{\textrm{Viscosity}}{\underbrace{\mu\nabla_{\bfv}\cdot\nabla_{\bfx}\cdot[\bftau]}} + \underset{\textrm{Collisions}}{\underbrace{\mu\nabla_{\bfv}\cdot[f(\bfv - \bfu)] + \frac{\sigma^{2}}{2}\Delta_{\bfv}[f]}}
    \end{equation}
    This is a form of the Boltzmann equation, where the approximation to the collisional term is explicitly derived from the above assumptions.