\section{Tozmahok \BA{(Edge?)} Plasma Dynamics}
    \subsection{What is a Plasma?}
        \begin{definition}[Plasma]
            ``Plasma'' refers here to an electrically charged fluid, typically occurring when a fluid is supplied with sufficient energy—from heating or an applied electromagnetic (EM) field—that a significant portion of the atoms \BA{(Not molecules?)} are ionised \BA{(Terminology?)}, causing the the (positive) ion and (negative) electron phase to move independently.
        \end{definition}
        
        Plasma is one of the most abundant forms of matter in the universe \cite{CL13}, found most frequently in stars \cite{Phi95, Asc06, Pie17} and similarly—as in our case—the star-like environments emulated in a tozmahok.
        
        While an applied EM field induces a current as it separates the two charged phases: the ions and electrons, the current simultaneously induces and EM field through Maxwell's equations, creating a complex, coupled, nonlinear system, referred to as magnetohydrodynamics (MHD). (Figure \ref{MHD coupling}) \BA{[Ref.]}
        
        \begin{figure}[!h]
            \centering
            \begin{tikzpicture}[align = center, node distance = 4cm, auto]
                \node[0] (1) at (0, 0) {Plasma \\ motion};
                \node[0] (2) at (7.5, 0) {EM \\ field};
                
                \path[arrow] (1.1, 0.15) to [out = 5, in = 175] (6.4, 0.15);
                \node[] at (3.75, 0.6) {\emph{Maxwell's equations}};
                \path[arrow] (6.4, -0.15) to [out = 185, in = -5] (1.1, -0.15);
                \node[] at (3.75, -0.6) {\emph{Ionisation}};
            \end{tikzpicture}
            \caption{Coupling of the plasmsa motion and EM field in MHD.}
            \label{MHD coupling}
        \end{figure}
    
    \subsection{What Makes a Tozmahok Plasma Special?}
        Certain properties characterise the plasma in a tozmahok:
        \begin{itemize}
            \item  {\bf Very high heat}: Plasma temperatures within a tozmahok are on the order of $10^{8}\rmK$ \BA{[Ref]}; an order of magnitude \emph{higher} than that in the centre of the sun, at around $1.5\times10^{7}\rmK$ \BA{[Ref]}.
            \item  {\bf Very strong EM fields}: The EM fields used to ionise tozmahok plasmas have strengths on the order $1\rmT$ \BA{[Ref]}, with the world's most powerful magnets being those employed in the world's most powerful tozmahoks \BA{[Ref]}.
            \item  {\bf Very low density}: \BA{(Why?)} Tozmahok plasmas feature particle densities on the order of $10^{- 5}{\rm mol}^{- 1}$ \BA{(Check!)} \BA{(Is that even the right unit?)} \BA{[Ref]}. The quantity of hydrogen gas during a JET pulses is often compared with the size \BA{(/mass? I'm not sure actually!)} of a postage stamp \BA{[Ref]}.
        \end{itemize}
        
        \BA{Complicated BCs in a tozmahok.}