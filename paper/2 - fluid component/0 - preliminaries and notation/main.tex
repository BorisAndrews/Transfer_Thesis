\section*{Preliminaries and Notation}
    \BA{Introduction.}

    Let $\bfOmega  \subset  \bbR^{3}$ denote a \BA{(sufficiently smooth/compact etc.)} domain, with boundary $\bfGamma  :=  \partial\bfOmega$, and denote the outward-pointing normal on $\bfGamma$ as $\bfn$. Supported on $\bfOmega$, denote the $L^{2}$ inner product on a domain $\bfS$ (e.g. $\bfOmega$, $\bfGamma$) $\langle-, -\rangle_{\bfS}$ or $(\langle-, -\rangle)|_{\bfS}$,
    \begin{align}
        \langle u, v\rangle_{\bfS}  :=  \int_{\bfS}uv,  &&
        \text{or}  &&
        \langle \bfu, \bfv\rangle_{\bfS}  :=  \int_{\bfS}\bfu\cdot\bfv,  &&
        \text{or}  &&
        \langle \bfu, \bfv\rangle_{\bfS}  :=  \int_{\bfS}\bfu:\bfv,  &&
        \text{etc.}
    \end{align}
    with associated $L^{2}$ norm on $\bfS$
    \begin{equation}
        \|u\|_{\bfS}  :=  \langle u, u\rangle_{\bfS}^{\frac{1}{2}}
    \end{equation}
    
    Define the Hilbert spaces: \BA{(Delete any of these that I don't use later.)}
    \begin{align}
        L^{2}  &:=  \left\{u \in L^{1}_{\rm loc} : \|u\|_{\bfOmega} < \infty\right\}  \\
        \bfH(\rmdiv)  &:=  \left\{\bfu \in \left(L^{1}_{\rm loc}\right)^{3} : \nabla\cdot\bfu \in L^{2}\right\}  \\
        \bfH(\bfcurl)  &:=  \left\{\bfu \in \left(L^{1}_{\rm loc}\right)^{3} : \nabla\wedge\bfu \in \left(L^{2}\right)^{3}\right\}  \\
        H^{1}  &:=  \left\{u \in L^{1}_{\rm loc} : \nabla u \in \left(L^{2}\right)^{3}\right\}
    \end{align}
    These spaces form a complex \BA{([Ref])}
    \begin{center}\begin{tikzpicture}[align = center, node distance = 4cm, auto]
        \node (H1)    at (0,    0) {$H^{1}$};
        \node (Hcurl) at (3.5,  0) {$\bfH(\bfcurl)$};
        \node (Hdiv)  at (7,    0) {$\bfH(\rmdiv)$};
        \node (L2)    at (10.5, 0) {$L^{2}$};

        \draw[->] (H1) -- (Hcurl) node[above, midway] {$\bfgrad$};
        \draw[->] (Hcurl) -- (Hdiv) node[above, midway] {$\bfcurl$};
        \draw[->] (Hdiv) -- (L2) node[above, midway] {$\rmdiv$};
    \end{tikzpicture}\end{center}
    Similarly, define the Hilbert spaces:
    \begin{align}
        L^{2}_{0}  &:=  \left\{u \in L^{2} : \int_{\bfOmega}u  = 
         0\right\}  \\
        \bfH_{0}(\rmdiv)  &:=  \left\{\bfu \in \bfH(\rmdiv) : \bfu\cdot\bfn = 0|_{\bfGamma}\right\}  \\
        \bfH_{0}(\bfcurl)  &:=  \left\{\bfu \in \bfH(\bfcurl) : \bfu\wedge\bfn = \bfzero|_{\bfGamma}\right\}  \\
        H^{1}_{0}  &:=  \left\{u \in H^{1} : u = 0|_{\bfGamma}\right\}
    \end{align}
    similarly forming a complex \BA{([Ref])}
    \begin{center}\begin{tikzpicture}[align = center, node distance = 4cm, auto]
        \node (H1)    at (0,    0) {$H^{1}_{0}$};
        \node (Hcurl) at (3.5,  0) {$\bfH_{0}(\bfcurl)$};
        \node (Hdiv)  at (7,    0) {$\bfH_{0}(\rmdiv)$};
        \node (L2)    at (10.5, 0) {$L^{2}_{0}$};

        \draw[->] (H1) -- (Hcurl) node[above, midway] {$\bfgrad$};
        \draw[->] (Hcurl) -- (Hdiv) node[above, midway] {$\bfcurl$};
        \draw[->] (Hdiv) -- (L2) node[above, midway] {$\rmdiv$};
    \end{tikzpicture}\end{center}
    
    Define the helicity, $H(\bfv)$, of a vector field $\bfv : \bfOmega \rightarrow \bbR^{3}$,
    \begin{equation}
        H(\bfv)  :=  \int_{\bfOmega}\bfv\cdot(\nabla\wedge\bfv)
    \end{equation}
