\section*{Preliminaries and Notation}
    \BA{Introduction.}

    Let $\bfOmega  \subset  \bbR^{3}$ denote a \BA{(sufficiently smooth/compact etc.)} domain, with boundary $\bfGamma  :=  \partial\bfOmega$, and denote the outward-pointing normal on $\bfGamma$ as $\bfn$. Supported on $\bfOmega$, denote the $L^{2}$ inner product on a domain $\bfS$ (e.g. $\bfOmega$, $\bfGamma$) $\langle-, -\rangle_{\bfS}$ or $(\langle-, -\rangle)|_{\bfS}$,
    \begin{align}
        \langle u, v\rangle_{\bfS}  :=  \int_{\bfS}uv,  &&
        \text{or}  &&
        \langle \bfu, \bfv\rangle_{\bfS}  :=  \int_{\bfS}\bfu\cdot\bfv,  &&
        \text{or}  &&
        \langle \bfu, \bfv\rangle_{\bfS}  :=  \int_{\bfS}\bfu:\bfv,  &&
        \text{etc.}
    \end{align}
    with associated $L^{2}$ norm on $\bfS$, $\|-\|_{\bfS}$ or $(\|-\|)|_{\bfS}$,
    \begin{equation}
        \|u\|_{\bfS}  :=  \langle u, u\rangle_{\bfS}^{\frac{1}{2}}
    \end{equation}
    
    Define the helicity, $\calH(\bfv)$, of a vector field $\bfv : \bfOmega \rightarrow \bbR^{3}$, \BA{(Maybe move this into the ``Preserved Structures'' section.)}
    \begin{equation}
        \calH(\bfv)  :=  \int_{\bfOmega}\bfv\cdot(\nabla\wedge\bfv)
    \end{equation}
    
    \subsection*{Complexes}
        \paragraph*{3D}
            Define the Hilbert spaces on $\bfOmega$:
            \begin{align}
                L^{2}          &:=  \left\{u \in L^{1}_{\rm loc} : \|u\|_{\bfOmega} < \infty\right\}  \\
                \bfH(\rmdiv)   &:=  \left\{\bfu \in \left(L^{1}_{\rm loc}\right)^{3} : \nabla\cdot\bfu \in L^{2}\right\}  \\
                \bfH(\bfcurl)  &:=  \left\{\bfu \in \left(L^{1}_{\rm loc}\right)^{3} : \nabla\wedge\bfu \in \left(L^{2}\right)^{3}\right\}  \\
                H^{1}          &:=  \left\{u \in L^{1}_{\rm loc} : \nabla u \in \left(L^{2}\right)^{3}\right\}
            \end{align}
            These spaces form a complex of vector proxies for $H\Lambda^{\bullet}(\bfOmega)$ \BA{([Ref])}
            \begin{center}\begin{tikzpicture}[align = center, node distance = 4cm, auto]
                \node (H1)    at (0,    0) {$H^{1}$};
                \node (Hcurl) at (3.5,  0) {$\bfH(\bfcurl)$};
                \node (Hdiv)  at (7,    0) {$\bfH(\rmdiv)$};
                \node (L2)    at (10.5, 0) {$L^{2}$};

                \draw[->] (H1)    -- (Hcurl) node[above, midway] {$\bfgrad$};
                \draw[->] (Hcurl) -- (Hdiv)  node[above, midway] {$\bfcurl$};
                \draw[->] (Hdiv)  -- (L2)    node[above, midway] {$\rmdiv$};
            \end{tikzpicture}\end{center}
            Similarly, define the Hilbert spaces:
            \begin{align}
                L^{2}_{0}          &:=  \left\{u \in L^{2} : \int_{\bfOmega}u = 0\right\}  \\
                \bfH_{0}(\rmdiv)   &:=  \left\{\bfu \in \bfH(\rmdiv) : \bfu\cdot\bfn = 0|_{\bfGamma}\right\}  \\
                \bfH_{0}(\bfcurl)  &:=  \left\{\bfu \in \bfH(\bfcurl) : \bfu\wedge\bfn = \bfzero|_{\bfGamma}\right\}  \\
                H^{1}_{0}          &:=  \left\{u \in H^{1} : u = 0|_{\bfGamma}\right\}
            \end{align}
            These spaces similarly form a complex of vector proxies for $H\Lambda^{\bullet}_{0}(\bfOmega)$: \BA{([Ref])}
            \begin{center}\begin{tikzpicture}[align = center, node distance = 4cm, auto]
                \node (H1)    at (0,    0) {$H^{1}_{0}$};
                \node (Hcurl) at (3.5,  0) {$\bfH_{0}(\bfcurl)$};
                \node (Hdiv)  at (7,    0) {$\bfH_{0}(\rmdiv)$};
                \node (L2)    at (10.5, 0) {$L^{2}_{0}$};

                \draw[->] (H1)    -- (Hcurl) node[above, midway] {$\bfgrad$};
                \draw[->] (Hcurl) -- (Hdiv)  node[above, midway] {$\bfcurl$};
                \draw[->] (Hdiv)  -- (L2)    node[above, midway] {$\rmdiv$};
            \end{tikzpicture}\end{center}
        
        \paragraph*{4D}
            For a given $T  >  0$, define the Hilbert spaces on $\bfOmega\otimes[0, T]$: \BA{(Bit of a clash in notation here on the $L^{2}$'s.)}
            \begin{align}
                L^{2}  &:=  \left\{u \in L^{1}_{\rm loc} : \|u\|_{\bfOmega\otimes[0, T]} < \infty\right\}  \\
                \bfH(\partial_{t} + \rmdiv)  &:=  \left\{\begin{pmatrix} u \\ \bfv \end{pmatrix} \in \left(L^{1}_{\rm loc}\right)^{1 + 3} : \partial_{t}u + \nabla\cdot\bfv \in L^{2}\right\}  \\
                \bfH\left(\begin{matrix} * - \rmdiv \\ \bfcurl + \partial_{t} \end{matrix}\right)  &:=  \left\{\begin{pmatrix} \bfu \\ \bfv \end{pmatrix} \in \left(L^{1}_{\rm loc}\right)^{3 + 3} : \begin{pmatrix} - \nabla\cdot\bfv \\ \nabla\wedge\bfu + \partial_{t}\bfv \end{pmatrix} \in \left(L^{2}\right)^{1 + 3}\right\}  \\
                \bfH\left(\begin{matrix} \bfgrad + \partial_{t} \\ * - \bfcurl \end{matrix}\right)  &:=  \left\{\begin{pmatrix} u \\ \bfv \end{pmatrix} \in \left(L^{1}_{\rm loc}\right)^{1 + 3} : \begin{pmatrix} \nabla u + \partial_{t}\bfv \\ - \nabla\wedge\bfv \end{pmatrix} \in \left(L^{2}\right)^{3 + 3}\right\}  \\
                H^{1}  &:=  \left\{u \in L^{1}_{\rm loc} : \begin{pmatrix} \partial_{t}u \\ - \nabla u \end{pmatrix} \in \left(L^{2}\right)^{1 + 3}\right\}
            \end{align}
            These spaces form a complex of vector proxies for $H\Lambda^{\bullet}(\bfOmega)$ \BA{([Ref])}
            {\footnotesize \begin{center}\begin{tikzpicture}[align = center, node distance = 4cm, auto]
                \node (HL0) at (0,    0) {$H^{1}$};
                \node (HL1) at (3.5,  0) {$\bfH\left(\begin{matrix} \bfgrad + \partial_{t} \\ * - \bfcurl \end{matrix}\right)$};
                \node (HL2) at (8.5,  0) {$\bfH\left(\begin{matrix} * - \rmdiv \\ \bfcurl + \partial_{t} \end{matrix}\right)$};
                \node (HL3) at (13,   0) {$\bfH(\partial_{t} + \rmdiv)$};
                \node (HL4) at (16,   0) {$L^{2}$};

                \draw[->] (HL0) -- (HL1) node[above, midway] {$\left(\begin{matrix} \partial_{t} \\ - \bfgrad \end{matrix}\right)$};
                \draw[->] (HL1) -- (HL2) node[above, midway] {$\left(\begin{matrix} \bfgrad + \partial_{t} \\ * - \bfcurl \end{matrix}\right)$};
                \draw[->] (HL2) -- (HL3) node[above, midway] {$\left(\begin{matrix} * - \rmdiv \\ \bfcurl + \partial_{t} \end{matrix}\right)$};
                \draw[->] (HL3) -- (HL4) node[above, midway] {$\partial_{t} + \rmdiv$};
            \end{tikzpicture}\end{center}}
            Similarly, define the Hilbert spaces:
            \begin{align}
                L^{2}_{0}  &:=  \left\{u \in L^{2} : \int_{\bfOmega\otimes[0, T]}u = 0\right\}  \\
                \bfH_{0}(\partial_{t} + \rmdiv)  &:=  \left\{\begin{pmatrix} u \\ \bfv \end{pmatrix} \in \bfH(\partial_{t} + \rmdiv) : \bfv\cdot\bfn|_{\bfGamma\otimes[0, T]}, u|_{\bfOmega\otimes\{0, T\}} = 0\right\}  \\
                \bfH_{0}\left(\begin{matrix} * - \rmdiv \\ \bfcurl + \partial_{t} \end{matrix}\right)  &:=  \left\{\begin{pmatrix} \bfu \\ \bfv \end{pmatrix} \in \bfH\left(\begin{matrix} * - \rmdiv \\ \bfcurl + \partial_{t} \end{matrix}\right) : \begin{matrix} \bfv\cdot\bfn|_{\bfGamma\otimes[0, T]} = 0 \\ \bfu\wedge\bfn|_{\bfOmega\otimes\{0, T\}}, \bfv|_{\bfGamma\otimes[0, T]} = \bfzero \end{matrix}\right\}  \\
                \bfH_{0}\left(\begin{matrix} \bfgrad + \partial_{t} \\ * - \bfcurl \end{matrix}\right)  &:=  \left\{\begin{pmatrix} u \\ \bfv \end{pmatrix} \in \bfH\left(\begin{matrix} \bfgrad + \partial_{t} \\ * - \bfcurl \end{matrix}\right) : \begin{matrix} u\bfn|_{\bfGamma\otimes[0, T]}, \bfv|_{\bfOmega\otimes\{0, T\}} = \bfzero \\ \bfv\wedge\bfn|_{\bfGamma\otimes[0, T]} = \bfzero \end{matrix}\right\}  \\
                H^{1}_{0}  &:=  \left\{u \in H^{1} : \begin{matrix} u = 0|_{\bfOmega\otimes\{0, T\}} \\ u = 0|_{\bfGamma\otimes[0, T]} \end{matrix}\right\}
            \end{align}
            These spaces similarly form a complex of vector proxies for $H\Lambda^{\bullet}_{0}(\bfOmega)$: \BA{([Ref])}
            {\footnotesize \begin{center}\begin{tikzpicture}[align = center, node distance = 4cm, auto]
                \node (HL0) at (0,    0) {$H^{1}_{0}$};
                \node (HL1) at (3.5,  0) {$\bfH_{0}\left(\begin{matrix} \bfgrad + \partial_{t} \\ * - \bfcurl \end{matrix}\right)$};
                \node (HL2) at (8.5,  0) {$\bfH_{0}\left(\begin{matrix} * - \rmdiv \\ \bfcurl + \partial_{t} \end{matrix}\right)$};
                \node (HL3) at (13,   0) {$\bfH_{0}(\partial_{t} + \rmdiv)$};
                \node (HL4) at (16,   0) {$L^{2}_{0}$};

                \draw[->] (HL0) -- (HL1) node[above, midway] {$\left(\begin{matrix} \partial_{t} \\ - \bfgrad \end{matrix}\right)$};
                \draw[->] (HL1) -- (HL2) node[above, midway] {$\left(\begin{matrix} \bfgrad + \partial_{t} \\ * - \bfcurl \end{matrix}\right)$};
                \draw[->] (HL2) -- (HL3) node[above, midway] {$\left(\begin{matrix} * - \rmdiv \\ \bfcurl + \partial_{t} \end{matrix}\right)$};
                \draw[->] (HL3) -- (HL4) node[above, midway] {$\partial_{t} + \rmdiv$};
            \end{tikzpicture}\end{center}}
