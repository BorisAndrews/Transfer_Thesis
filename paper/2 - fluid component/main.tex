\chapter{Fluid Component}
    \BA{Introduction.}

    \BA{(This is a little rough. I will give a full derivation of this model, but that'll be in the previous section. For now, I just want to have the system written down.)} Without the $(\delta f_{s})_{s}$ corrections, the fluid component of the system takes the following form, up to leading order: \BA{(Also, too many $\mu$'s! And that's not to mention the $p$'s and $\bfp$'s! Might change the $\mu$'s in the drift terms to $\theta$s, akin to the drift terms in the Ornstein–Uhlenbeck process.)}
    \begin{align}
        \partial_{t}\rho_{M} + \nabla\cdot\bfp  &=  0  \\
        \partial_{t}\rho_{C} + \nabla\cdot\bfj  &=  0  \\
        \rho_{M}\partial_{t}\bfu + \bfp\cdot\nabla\bfu  &=  - \nabla p + (\rho_{C}\bfE + \bfj\wedge\bfB) + \nu\nabla\cdot[\rho_{M}\bftau]  \\
        \partial_{t}p + \nabla\cdot[p\bfu]  &=  - p\nabla\cdot\bfu + \nu\rho_{M}\bftau:\nabla_{\rms}\bfu + \frac{\mu_{+-}}{q_{+}q_{-}}\|\bfj\|^{2} + \frac{\kappa k_{B}}{m_{+}}\nabla\cdot[\rho_{M}\nabla\theta]
    \end{align}
    \BA{(Clarify what I mean by $*^{\otimes 2}$. I also need to move this note now I've moved the $*^{\otimes 2}$ haha.)} \BA{(Also the $\partial_{t}p$ terms should have, like, a $\frac{3}{2}$ term, or something to that effect, before it, but that's something I'll sort when I derive the exact discretization. It doesn't change tthe analysis.)} alongside the equilibrium relation for $\bfj$,
    \begin{equation}
        \bfzero  =  \frac{\mu_{+-}}{q_{+}q_{-}}\bfj - (\bfE + \bfu\wedge\bfB) + \BA{\rm const.}\bfj\wedge\bfB
    \end{equation}
    where $\bfp$, $\bfu$ and $p$, $\rho_{M}$ are related by
    \begin{align}
        \bfp  =  \rho_{M}\bfu,  &&
        m_{+}p  =  k_{B}\rho_{M}\theta
    \end{align}
    coupled with Maxwell's equations:
    \begin{align*}
        \partial_{t}\bfE  &=  c^{2}\nabla\wedge\bfB - \frac{1}{\varepsilon_{0}}\bfj,  &
        \partial_{t}\bfB  &=  - \nabla\wedge\bfE  \\
        \nabla\cdot\bfE  &=  \frac{1}{\varepsilon_{0}}\rho_{C},  &
        \nabla\cdot\bfB  &=  0
    \end{align*}
    where $c \approx 2.98\times 10^{8}\rmm\rms^{- 1}$ denotes the speed of light. \BA{(Move the definition for $c$ to earlier when I define Maxwell's equations.)} For conciseness, $\bftau(\bfu)$ denotes the (deviatoric) strain
    \begin{equation}\label{eqn:strain equation}
        \bftau(\bfu)  :=  2\nabla_{\rms}\bfu - \frac{2}{3}(\nabla\cdot\bfu)\bfI
    \end{equation}
    and $\nabla_{\rms}$ denotes the symmetric gradient,
    \begin{equation}
        \nabla_{\rms}\bfv  :=  \frac{1}{2}\left(\nabla\bfv + \nabla\bfv^{T}\right)
    \end{equation}

    To non-dimensionalize the above system, denote the scale of a variable $*$ as $\overline{*}$. Assume that $\overline{\rho_{M}}$, $\overline{\bfp}$, and $\overline{\bfB}$, are defined from the boundary conditions, and the scale of $\overline{\bfx}$ is defined by the domain. $\overline{\bfu}$, $\overline{\theta}$, $\overline{\bftau}$ are naturally given from their definitions as:
    \begin{align}
        \overline{\bfu}  =  \frac{\overline{\bfp}}{\overline{\rho_{M}}},  &&
        \overline{\theta}  =  \frac{m_{+}}{k_{B}}\cdot{\overline{\theta}\overline{\rho_{M}}},  &&
        \overline{\bftau}  :=  \frac{\overline{\bfp}}{\overline{\rho_{M}}\overline{\bfx}}
    \end{align}
    The remaining scales, $\overline{t}$, $\overline{\rho_{C}}$, $\overline{\bfj}$, $\overline{p}$, $\overline{\bfE}$, and $\overline{\bftau}$ are defined as:  \BA{(Don't like much of the phrasing or the format here. I feel the equations are a little hard to read.)}
    \begin{itemize}
        \item  $\overline{t}  :=  \frac{\overline{\rho_{M}}\overline{\bfx}}{\overline{\bfp}}$, working on \emph{convective} timescales.
        \item  $\overline{\bfj}  :=  \frac{\overline{\bfB}}{\mu_{0}\overline{\bfx}}$, as $\bfj$ is brought to the scale of $\nabla\wedge\bfB$.
        \item  $\overline{\rho_{C}}  :=  \frac{1}{\mu_{0}}\cdot\frac{\overline{\rho_{M}}\overline{\bfB}}{\overline{\bfx}\overline{\bfp}}  \left(=  \frac{\overline{\bfj}\overline{t}}{\overline{\bfx}}\right)$, as charge is induced by divergence in the current.
        \item  $\overline{p}  :=  \frac{\overline{\bfp}^{2}}{\overline{\rho_{M}}}$, such that kinetic and internal energy are on the same scale.
        \item  $\overline{\bfE}  :=  \frac{\overline{\bfB}\overline{\bfp}}{\overline{\rho_{M}}}  \left(=  \frac{\overline{\bfB}\overline{\bfx}}{\overline{t}}\right)$, as the curl in electric field is brought to balance the change in the magnetic field.
    \end{itemize}
    Defining the dimensionless constants in Figure \ref{fig:dimensionless quantities}, this takes the non-dimensionalized form:
    \begin{align}
        \partial_{t}\rho_{M} + \nabla\cdot\bfp  &=  0  \\
        \partial_{t}\rho_{C} + \nabla\cdot\bfj  &=  0  \\
        \partial_{t}\bfp + \bfp\cdot\nabla\bfu  &=  - \nabla p + \frac{2}{\beta}(\rho_{C}\bfE + \bfj\wedge\bfB) + \frac{1}{\rmRef}\nabla\cdot[\rho_{M}\bftau]  \\
        \partial_{t}p + \nabla\cdot[p\bfu]  &=  - p\nabla\cdot\bfu + \frac{1}{\rmRef}\rho_{M}\bftau:\nabla_{\rms}\bfu + \frac{2}{\beta\rmRem}\|\bfj\|^{2} + \frac{1}{\rmPe}\nabla\cdot[\rho_{M}\nabla\theta]
    \end{align}
    alongside the equilibrium relation for $\bfj$,
    \begin{equation}
        \bfzero  =  \frac{1}{\rmRem}\bfj - \left(\bfE + \bfu\wedge\bfB\right) + \rmRH\bfj\wedge\bfB
    \end{equation}
    identities for $\bfu$, $\theta$,
    \begin{align}
        \bfp  =  \rho_{M}\bfu,  &&
        p  =  \rho_{M}\theta
    \end{align}
    and the non-dimensionalized forms of Maxwell's equations:
    \begin{align*}
        \rmM^{2}\partial_{t}\bfE  &=  \nabla\wedge\bfB - \bfj,  &
        \partial_{t}\bfB  &=  - \nabla\wedge\bfE  \\
        \rmM^{2}\nabla\cdot\bfE  &=  \rho_{C},  &
        \nabla\cdot\bfB  &=  0
    \end{align*}
    
    \begin{figure}[!h]
        \begin{tabular}{ c c c c }
            Name  &  Symbol  &  Value  &  Ratio  \\
            \hline\hline
            Fluid Reynolds number  &  $\rmRef$  &  $\frac{1}{\nu}\cdot\frac{\overline{\bfx}\overline{\bfp}}{\overline{\rho_{M}}}$  &  Momentum (advection : diffusion)  \\
            Magnetic Reynolds number  &  $\rmRem$  &  $\frac{q_{+}q_{-}\mu_{0}}{\mu_{+-}}\cdot\frac{\overline{\bfx}\overline{\bfp}}{\overline{\rho_{M}}}$  &  Magnetic (advection : diffusion)  \\
            Péclet number  &  $\rmPe$  &  $\frac{1}{\kappa}\cdot\frac{\overline{\bfx}\overline{\bfp}}{\overline{\rho_{M}}}$  &  Pressure (advection : diffusion)  \\
            \hline
            Plasma beta  &  $\beta$  &  $2\mu_{0}\cdot\frac{\overline{\bfp}^{2}}{\overline{\rho_{M}}\overline{\bfB}^{2}}$  &  (Plasma : Magnetic) pressure  \\
            Hall number  &  $\rmRH$  &  \BA{$\frac{m_{+}}{q_{+}\mu_{0}}\frac{\overline{\bfB}^{2}}{\overline{\rho_{M}}\overline{\bfx}}$}  &  \BA{??}  \\
            (Light) Mach number  &  $\rmM$  &  $\frac{1}{c}\cdot\frac{\overline{\bfp}}{\overline{\rho_{M}}}$  &  (Plasma : Light) speed
        \end{tabular}
        \caption{Dimensionless quantities in the compressible Hall MHD system. \BA{(Need to re-derive and check the Hall number.)}}
        \label{fig:dimensionless quantities}
    \end{figure}

    On non-relativistic scale where $\rmM  \ll  1$, the quasi-neutral hypothesis is derived $\rho_{C}  \sim  0$. This is assumed frm here. This gives the following, final, quasi-neutral compressible Hall MHD system, where, with $\rho_{C}$ eliminated, $\rho_{M}$ is denoted as simply $\rho$:
    
    \line

    \paragraph*{Compressible Strong Formulation}
    
    \begin{align}
        \partial_{t}\rho  &=  - \nabla\cdot\bfp  \label{eqn:mass conservation}  \\
        \partial_{t}\bfp  &=  - \bfp\cdot\nabla\bfu - \nabla p + \frac{2}{\beta}\bfj\wedge\bfB + \frac{1}{\rmRef}\nabla\cdot[\rho\bftau]  \label{eqn:momentum conservation}  \\
        \partial_{t}p  &=  - \nabla\cdot[p\bfu] - p\nabla\cdot\bfu + \frac{1}{\rmRef}\rho\bftau:\nabla_{\rms}\bfu + \frac{2}{\beta\rmRem}\|\bfj\|^{2} + \frac{1}{\rmPe}\nabla\cdot[\rho\nabla\theta]  \label{eqn:energy conservation}  \\
        \bfzero  &=  \frac{1}{\rmRem}\bfj - \left(\bfE + \bfu\wedge\bfB\right) + \rmRH\bfj\wedge\bfB  \label{eqn:current identity}  \\
        \bfzero  &=  \bfp - \rho\bfu  \label{eqn:velocity identity}  \\
        0  &=  p - \rho\theta  \label{eqn:temperature identity}  \\
        \bfzero  &=  \bfj - \nabla\wedge\bfB  \label{eqn:Ampère's law}  \\
        \partial_{t}\bfB  &=  - \nabla\wedge\bfE  \label{eqn:Faraday's law}  \\
        0  &=  \nabla\cdot\bfB  \label{eqn:Gauss's law}
    \end{align}

    \line

    Two notes on this system:
    \begin{itemize}
        \item  For transient models, Gauss's law, (\ref{eqn:Gauss's law}), need only be enforced on the initial conditions at time $t  =  0$, as it is enforced for $t  >  0$ by Faraday's law, (\ref{eqn:Faraday's law}).

        \item  In the ideal limit $\rmRef, \rmRem  =  \infty$, the viscous and Ohmic heating terms, $\frac{1}{\rmRef}\rho\bftau:\nabla_{\rms}\bfu$ and $\frac{2}{\beta\rmRem}\|\bfj\|^{2}$ are negligible. Despite considering turbulent, high-Reynolds systems, they are left in the system, as they give exact energy conservation for general $\rmRef$, $\rmRem$, giving a more rich structure for the analysis and discretization. 
    \end{itemize}

    This system resembles the \emph{incompressible} Hall MHD system presented in \cite{LHF22}, incorporating compressibility, a necessary factor for kinetic effects, through the inclusion of the energy equation, (\ref{eqn:energy conservation}), with the notation differing slightly in places. The system there can be written in the form:
    
    \line

    \paragraph*{Incompressible Strong Formulation}
    
    \begin{align}
        0  &=  \nabla\cdot\bfu  \label{eqn:incompressible mass conservation}  \\
        \partial_{t}\bfu  &=  - \bfu\cdot\nabla\bfu - \nabla p + \frac{2}{\beta}\bfj\wedge\bfB + \frac{1}{\rmRef}\Delta\bfu  \label{eqn:incompressible momentum conservation}  \\
        \bfzero  &=  \frac{1}{\rmRem}\bfj - \left(\bfE + \bfu\wedge\bfB\right) + \rmRH\bfj\wedge\bfB  \label{eqn:incompressible current identity}  \\
        \bfzero  &=  \bfj - \nabla\wedge\bfB  \label{eqn:incompressible Ampère's law}  \\
        \partial_{t}\bfB  &=  - \nabla\wedge\bfE  \label{eqn:incompressible Faraday's law}  \\
        0  &=  \nabla\cdot\bfB  \label{eqn:incompressible Gauss's law}
    \end{align}

    \line

    One can seek therefore to analyse, discretize and precondition this system through similar techniques.\footnote{Note, the model in \cite{LHF22} is written in terms of the velocity, $\bfu  :=  \frac{1}{\rho_{M}}\bfp$, instead of the momentum, $\bfp$, since by constant density, $\rho_{M}$, the two are functionally equivalent after non-dimensionalization. We choose to write the compressible system here in terms of ``conservation quantities'' through the momentum, $\bfp$, to ensure the mass conservation equation, (\ref{eqn:mass conservation}), is linear, such that we can ensure it holds \emph{exactly} in the numerical discretization. \\ The system there is also written in terms of a coupling constant, $\rmS  :=  \frac{2}{\beta}$. We choose the dimensionless parameter, $\beta$, to align more with the pre-existing physics literature. \BA{([Ref])}}
    
    \BA{Note to self, for collision frequency $\nu_{\rm Coll}$,
    \begin{equation}
        m_{+}m_{-}\nu_{\rm Coll}  =  \mu_{+-}\rho_{M}
    \end{equation}}

    
    \renewcommand{\arraystretch}{1.5}
\addbibresource{references.bib}
\def\contra{
    \tikz[baseline, x=0.22em, y=0.22em, line width=0.032em]
    \draw (0,2.83)--(2.83,0) (0.71,3.54)--(3.54,0.71) (0,0.71)--(2.83,3.54) (0.71,0)--(3.54,2.83);
}

    \renewcommand{\arraystretch}{1.5}
\addbibresource{references.bib}
\def\contra{
    \tikz[baseline, x=0.22em, y=0.22em, line width=0.032em]
    \draw (0,2.83)--(2.83,0) (0.71,3.54)--(3.54,0.71) (0,0.71)--(2.83,3.54) (0.71,0)--(3.54,2.83);
}

    \renewcommand{\arraystretch}{1.5}
\addbibresource{references.bib}
\def\contra{
    \tikz[baseline, x=0.22em, y=0.22em, line width=0.032em]
    \draw (0,2.83)--(2.83,0) (0.71,3.54)--(3.54,0.71) (0,0.71)--(2.83,3.54) (0.71,0)--(3.54,2.83);
}

    \renewcommand{\arraystretch}{1.5}
\addbibresource{references.bib}
\def\contra{
    \tikz[baseline, x=0.22em, y=0.22em, line width=0.032em]
    \draw (0,2.83)--(2.83,0) (0.71,3.54)--(3.54,0.71) (0,0.71)--(2.83,3.54) (0.71,0)--(3.54,2.83);
}


    
    \section*{Summary}
        \BA{Summary.}
    