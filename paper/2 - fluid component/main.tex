\chapter{Fluid Component}
    \BA{Introduction.}

    \BA{(This is a little rough. I will give a full derivation of this model, but that'll be in the previous section. For now, I just want to have the system written down.)} Without the $(\delta f_{s})_{s}$ corrections, the fluid component of the system takes the following form, up to leading order: \BA{(Also, too many $\mu$'s! And that's not to mention the $p$'s and $\bfp$'s! Might change the $\mu$'s in the drift terms to $\theta$s, akin to the drift terms in the Ornstein–Uhlenbeck process.)}
    \begin{align}
        \partial_{t}\rho_{M} + \nabla\cdot\bfp  &=  0  \\
        \partial_{t}\rho_{C} + \nabla\cdot\bfj  &=  0  \\
        \partial_{t}\bfp + \nabla\cdot\left[\frac{1}{\rho_{M}}\bfp^{\otimes 2}\right]  &=  - \nabla p + (\rho_{C}\bfE + \bfj\wedge\bfB) + \nu\nabla\cdot[\rho_{M}\bfsigma]  \\
        \partial_{t}p + \nabla\cdot\left[\frac{p}{\rho_{M}}\bfp\right]  &=  - p\nabla\cdot\left[\frac{1}{\rho_{M}}\bfp\right] + \nu\rho_{M}\bfsigma:\nabla\left[\frac{1}{\rho_{M}}\bfp\right] + \frac{\mu_{+-}}{q_{+}q_{-}}\|\bfj\|^{2} + \kappa\Delta\left[\frac{p}{\rho_{M}}\right]
    \end{align}
    where $\nu$ is the (kinematic) viscosity, and $\bfsigma$ is the (deviatoric) strain
    \begin{equation}\label{eqn:strain equation}
        \bfsigma  :=  2\sym\left(\nabla\left[\frac{1}{\rho_{M}}\bfp\right]\right) - \frac{2}{3}\nabla\cdot\left[\frac{1}{\rho_{M}}\bfp\right]\bfI
    \end{equation}
    alongside the equilibrium relation for $\bfj$,
    \begin{equation}
        \bfzero  =  \frac{\mu_{+-}}{q_{+}q_{-}}\bfj - \left(\bfE + \frac{1}{\rho_{M}}\bfp\wedge\bfB\right) + \BA{\rm const.}\bfj\wedge\bfB
    \end{equation}
    This is coupled with Maxwell's equations:
    \begin{align*}
        \partial_{t}\bfE  &=  c^{2}\nabla\wedge\bfB - \frac{1}{\varepsilon_{0}}\bfj,  &
        \partial_{t}\bfB  &=  - \nabla\wedge\bfE  \\
        \nabla\cdot\bfE  &=  \frac{1}{\varepsilon_{0}}\rho_{C},  &
        \nabla\cdot\bfB  &=  0
    \end{align*}
    where $c \approx 2.98\times 10^{8}\rmm\rms^{- 1}$ denotes the speed of light.

    To non-dimensionalize the above system, denote the scale of a variable $*$ as $\overline{*}$. Assume that $\overline{\rho_{M}}$, $\overline{\bfp}$, and $\overline{\bfB}$, are defined from the boundary conditions, and the scale of $\overline{\bfx}$ is defined by the domain. The remaining scales, $\overline{t}$, $\overline{\rho_{C}}$, $\overline{\bfj}$, $\overline{p}$, $\overline{\bfE}$, and $\overline{\bfsigma}$ are defined as:  \BA{(Don't like much of the phrasing or the format here. I feel the equations are a little hard to read.)}
    \begin{itemize}
        \item  $\overline{t}  :=  \frac{\overline{\rho_{M}}\overline{\bfx}}{\overline{\bfp}}$, such that we are working on \emph{convective} timescales.
        \item  $\overline{\bfj}  :=  \frac{\overline{\bfB}}{\mu_{0}\overline{\bfx}}$, as $\bfj$ is brought to the scale of $\nabla\wedge\bfB$.
        \item  $\overline{\rho_{C}}  :=  \frac{\overline{\rho_{M}}\overline{\bfB}}{\mu_{0}\overline{\bfx}\overline{\bfp}}  \left(=  \frac{\overline{\bfj}\overline{t}}{\overline{\bfx}}\right)$, as charge is induced by divergence in the current.
        \item  $\overline{p}  :=  \frac{\overline{\bfp}^{2}}{\overline{\rho_{M}}}$, such that kinetic and internal energy are on the same scale.
        \item  $\overline{\bfE}  :=  \frac{\overline{\bfB}\overline{\bfp}}{\overline{\rho_{M}}}  \left(=  \frac{\overline{\bfB}\overline{\bfx}}{\overline{t}}\right)$, as the curl in electric field is brought to balance the change in the magnetic field.
        \item  $\overline{\bfsigma}  :=  \frac{\oveline{\bfp}}{\overline{\rho_{M}}\overline{\bfx}}$, from its definition.
    \end{itemize}
    Defining the dimensionless constants in Figure \ref{fig:dimensionless quantities}, this takes the non-dimensionalized form:
    {\small \begin{align}
        \partial_{t}\rho_{M} + \nabla\cdot\bfp  &=  0  \\
        \partial_{t}\rho_{C} + \nabla\cdot\bfj  &=  0  \\
        \partial_{t}\bfp + \nabla\cdot\left[\frac{1}{\rho_{M}}\bfp^{\otimes 2}\right]  &=  \nabla p + \frac{2}{\beta}(\rho_{C}\bfE + \bfj\wedge\bfB) + \frac{1}{\rmRef}\nabla\cdot[\rho_{M}\bfsigma]  \\
        \partial_{t}p + \nabla\cdot\left[\frac{p}{\rho_{M}}\bfp\right]  &=  - p\nabla\cdot\left[\frac{1}{\rho_{M}}\bfp\right] + \frac{1}{\rmRef}\rho_{M}\bfsigma:\nabla\left[\frac{1}{\rho_{M}}\bfp\right] + \frac{2}{\beta\rmRem}\|\bfj\|^{2} + \frac{1}{\rmPe}\Delta\left[\frac{p}{\rho_{M}}\right]
    \end{align}}
    alongside the equilibrium relation for $\bfj$,
    \begin{equation}\label{eqn:current identity}
        \bfzero  =  \frac{1}{\rmRem}\bfj - \left(\bfE + \frac{1}{\rho_{M}}\bfp\wedge\bfB\right) + \rmRH\bfj\wedge\bfB
    \end{equation}
    and the non-dimensionalized forms of Maxwell's equations:
    \begin{align*}
        \rmM^{2}\partial_{t}\bfE  &=  \nabla\wedge\bfB - \bfj,  &
        \partial_{t}\bfB  &=  - \nabla\wedge\bfE  \\
        \rmM^{2}\nabla\cdot\bfE  &=  \rho_{C},  &
        \nabla\cdot\bfB  &=  0
    \end{align*}
    The equation for $\bfsigma$, equation (\ref{eqn:strain equation}), remains unchanged.
    
    \begin{figure}[!h]
        \begin{tabular}{ c c c c }
            Name  &  Symbol  &  Value  &  Ratio  \\
            \hline\hline
            Fluid Reynolds number  &  $\rmRef$  &  $\frac{\overline{\rho_{M}}\overline{\bfx}\overline{\bfp}}{\nu}$  &  Momentum (advection : diffusion)  \\
            Magnetic Reynolds number  &  $\rmRem$  &  $\frac{q_{+}q_{-}\mu_{0}\overline{\bfx}\overline{\bfp}}{\mu_{+-}\overline{\rho_{M}}}$  &  Magnetic (advection : diffusion)  \\
            Péclet number  &  $\rmPe$  &  $\frac{\overline{\bfx}\overline{\bfp}}{\kappa}$  &  Pressure (advection : diffusion)  \\
            \hline
            Plasma beta  &  $\beta$  &  $\frac{2\mu_{0}\overline{\bfp}^{2}}{\overline{\rho_{M}}\overline{\bfB}^{2}}$  &  (Plasma : Magnetic) pressure  \\
            Hall number  &  $\rmRH$  &  \BA{$\frac{m_{+}\overline{\bfB}^{2}}{q_{+}\mu_{0}\overline{\rho_{M}}\overline{\bfx}}$}  &  \BA{??}  \\
            (Light) Mach number  &  $\rmM$  &  $\frac{\overline{\bfp}}{\overline{\rho_{M}}c}$  &  (Plasma : Light) speed
        \end{tabular}
        \caption{Dimensionless quantities in the compressible Hall MHD system. \BA{(Need to re-derive and check the Hall number.)}}
        \label{fig:dimensionless quantities}
    \end{figure}

    Assuming $\rmM  \ll  1$, we derive the quasi-neutral hypothesis $\rho_{C}  \sim  0$, with the system of equations:
    {\small \begin{align}
        \partial_{t}\rho_{M} + \nabla\cdot\bfp  &=  0  \label{eqn:mass conservation} 
         \\
        \partial_{t}\bfp + \nabla\cdot\left[\frac{1}{\rho_{M}}\bfp^{\otimes 2}\right]  &=  - \nabla p + \frac{2}{\beta}\bfj\wedge\bfB + \frac{1}{\rmRef}\nabla\cdot[\rho_{M}\bfsigma]  \label{eqn:momentum conservation}  \\
        \partial_{t}p + \nabla\cdot\left[\frac{p}{\rho_{M}}\bfp\right]  &=  - p\nabla\cdot\left[\frac{1}{\rho_{M}}\bfp\right] + \frac{1}{\rmRef}\rho_{M}\bfsigma:\nabla\left[\frac{1}{\rho_{M}}\bfp\right] + \frac{2}{\beta\rmRem}\|\bfj\|^{2} + \frac{1}{\rmPe}\Delta\left[\frac{p}{\rho_{M}}\right]  \label{eqn:energy conservation}
    \end{align}}
    coupled with Maxwell's equations:
    \begin{align}
        \bfzero  &=  \nabla\wedge\bfB - \bfj  \label{eqn:Ampère's law}  \\
        \partial_{t}\bfB  &=  - \nabla\wedge\bfE  \label{eqn:Faraday's law}  \\
        \nabla\cdot\bfB  &=  0  \label{eqn:Gauss's law}
    \end{align}
    The equilibrium relation for $\bfj$, (\ref{eqn:current identity}), remains unchanged. For transient models, Gauss's law, (\ref{eqn:Gauss's law}), is discarded from the system, as it ensured by Faraday's law, (\ref{eqn:Faraday's law}).

    This system resembles the \emph{incompressible} Hall MHD system presented in \cite{LHF22}, incorporating compressibility—a necessary factor for kinetic effects—through the introduction of the variable $\rho_{M}$, and the energy equation, (\ref{eqn:energy conservation}). As such, one can seek to analyse, discretize and precondition this system through similar techniques, although the notation here differs slightly in places. These equations are also written in terms of ``conservation variables'' via the momentum $\bfp$, instead of the velocity $\bfu  :=  \frac{1}{\rho_{M}}\bfp$, such that the mass conservation equation (\ref{eqn:mass conservation}) is linear, a fact that will be exploited when constructing the augmented Lagrangian preconditioner.

    From here, the quasi-neutral hypothesis shall be assumed. With $\rho_{C}$ eliminated, $\rho_{M}$ shall be denoted as simply $\rho$.
    
    \BA{Note to self, for collision frequency $\nu_{\rm Coll}$,
    \begin{equation}
        m_{+}m_{-}\nu_{\rm Coll}  =  \mu_{+-}\rho_{M}
    \end{equation}}

    
    \renewcommand{\arraystretch}{1.5}
\addbibresource{references.bib}
\def\contra{
    \tikz[baseline, x=0.22em, y=0.22em, line width=0.032em]
    \draw (0,2.83)--(2.83,0) (0.71,3.54)--(3.54,0.71) (0,0.71)--(2.83,3.54) (0.71,0)--(3.54,2.83);
}

    \renewcommand{\arraystretch}{1.5}
\addbibresource{references.bib}
\def\contra{
    \tikz[baseline, x=0.22em, y=0.22em, line width=0.032em]
    \draw (0,2.83)--(2.83,0) (0.71,3.54)--(3.54,0.71) (0,0.71)--(2.83,3.54) (0.71,0)--(3.54,2.83);
}

    \renewcommand{\arraystretch}{1.5}
\addbibresource{references.bib}
\def\contra{
    \tikz[baseline, x=0.22em, y=0.22em, line width=0.032em]
    \draw (0,2.83)--(2.83,0) (0.71,3.54)--(3.54,0.71) (0,0.71)--(2.83,3.54) (0.71,0)--(3.54,2.83);
}

    \renewcommand{\arraystretch}{1.5}
\addbibresource{references.bib}
\def\contra{
    \tikz[baseline, x=0.22em, y=0.22em, line width=0.032em]
    \draw (0,2.83)--(2.83,0) (0.71,3.54)--(3.54,0.71) (0,0.71)--(2.83,3.54) (0.71,0)--(3.54,2.83);
}


    
    \section*{Summary}
        \BA{Summary.}
    