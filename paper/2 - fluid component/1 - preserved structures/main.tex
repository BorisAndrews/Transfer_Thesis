\section{Preserved Structures}
    \BA{Introduction.}
    
    Consider first those quantities that are conserved by the transient system. The ability of the discretization to represent the system's physical behavior can be improved by ensuring that those properties that are conserved in the continuous case are \emph{also} conserved in the discretization. \BA{[Ref, ...]}
    
    \cite{LHF22} considers conservation of the following 3 quantities in the incompressible case: \BA{(Oops I've never defined $\bfA$! That should probably be in the introduction...)}
    \begin{center}\begin{tabular}{ c | r l }
        Properties  &  Symbol  &  Definition  \\
        \hline\hline
        Energy  &  $\rmE$  &  $\frac{1}{2}\|\bfp\|_{\bfOmega}^{2} + \frac{1}{\beta}\|\bfB\|_{\bfOmega}^{2}$  \\
        Magnetic helicity  &  $\rmH_{\rmM}$  &  $H(\bfA)$  \\
        Hybrid helicity  &  $\rmH_{\rmH}$  &  $\int_{\bfOmega}(a\bfA + \bfp)\cdot(b\bfB + \nabla\wedge\bfp)$
    \end{tabular}\end{center}
    where $a$, $b$ satisfy the relation $a + b  =  \frac{4}{\beta\rmRH}$. \BA{(What do these represent \emph{physically}? Diagrams!)} Taking the derivatives of these quantities over time (still in the incompressible system) gives \BA{(Proofs in appendix?)}
    \begin{align}
        \begin{split}
            \frac{d\rmE}{dt}  &=  \oint_{\bfGamma}\left[- \left(\frac{1}{2}\|\bfp\|^{2}\bfI + p\bfI - \frac{1}{\rmRef}\nabla\bfp^{T}\right)\bfp + \frac{1}{\beta}\bfB\wedge\bfE\right]\cdot\bfn  \\
            &\;\;\;\;\;\;\;\;\;\;\;\;\;\;\;\;- \left.\left(\frac{1}{\rmRef}\|\nabla\bfp\|^{2} + \frac{2}{\beta\rmRem}\|\bfj\|^{2}\right)\right|_{\bfOmega}
        \end{split}  \\
        \frac{d\rmH_{\rmM}}{dt}  &=  \oint_{\bfGamma}[\bfA\wedge(\bfE - \nabla\varphi)]\cdot\bfn - \frac{2}{\rmRem}H(\bfB)  \\
        \begin{split}
            \frac{d\rmH_{\rmH}}{dt}  &=  \oint_{\bfGamma}\left[ab\bfA\wedge(\bfE - \nabla\varphi) + (a + b)\left(\bfp\wedge\bfE - p\bfB - \frac{1}{2}\|\bfp\|^{2}\bfB\right)\right.  \\
            &\;\;\;\;\;\;\;\;\;\;\;\;\;\;\;\;\;\;\;\;\;\;\;\;\;\;\;\;\;\;\;\;\left.+ a\bfA\wedge\bfp + \left(\partial_{t}\bfp\wedge\bfp - \|\bfp\|^{2}\bfomega - 2p\bfomega + \frac{1}{\rmRef}\bfB\wedge\bfomega\right)\right]\cdot\bfn  \\
            &\;\;\;\;\;\;\;\;\;\;\;\;\;\;\;\;- \left[\frac{2ab}{\rmRem}H(\bfB) + (a + b)\left(\frac{1}{\rmRem} + \frac{1}{\rmRef}\right)\langle\bfj, \bfomega\rangle_{\bfOmega} + \frac{2}{\rmRef}H(\bfomega)\right]
        \end{split}
    \end{align}
    noting $\partial_{t}\bfA  =  - \nabla\varphi - \bfE$, where we define the vorticity $\bfomega  :=  \nabla\wedge\bfp$. Thus, for boundary conditions on $\bfGamma$, \BA{(Hmmmmm, haven't defined $\varphi$ either...)}
    \begin{align}
        \bfzero  =  \bfp,  &&
        0  =  \bfB\cdot\bfn,  &&
        \bfzero  =  \bfE\wedge\bfn \text{ or } \bfB\wedge\bfn,  &&
        \bfzero  =  (\bfE - \nabla\varphi)\wedge\bfn \text{ or } \bfA\wedge\bfn
    \end{align}
    $\frac{d\rmE}{dt}  \leq  0$, and moreover, as shown in \cite{LHF22}, in the ideal limit for $\rmRef, \rmRem  =  \infty$,
    \begin{equation}
        \frac{d\rmE}{dt}, \frac{d\rmH_{\rmM}}{dt}, \frac{d\rmH_{\rmH}}{dt}  =  0
    \end{equation}

    \BA{\cite{LHF22} showed that certain discretizations preserve these properties.}

    For the \emph{compressible} case, one can consider the energy, $\rmE$, redefined to incorporate the internal energy, and magnetic helicity, $\rmH_{\rmM}$
    \begin{center}\begin{tabular}{ c | r l }
        Properties  &  Symbol  &  Definition  \\
        \hline\hline
        Energy  &  $\rmE$  &  $\frac{1}{2}\|\bfp\|_{\bfOmega}^{2} + \frac{1}{\beta}\|\bfB\|_{\bfOmega}^{2} + \int_{\bfOmega}p$  \\
        Magnetic helicity  &  $\rmH_{\rmM}$  &  $H(\bfA)$
    \end{tabular}\end{center}
    \BA{(Not sure about the hybrid helicity, $\rmH_{\rmH}$. I can't get anything to work...)} The time derivative for the energy then only includes boundary contributions, \BA{(Proof in appendix?)}
    \begin{equation}
        \frac{d\rmE}{dt}  =  \oint_{\bfGamma}\left[- \left(\frac{1}{2\rho}\|\bfp\|^{2}\bfI + \frac{p}{\rho}\bfI - \frac{1}{\rmRef}\bfsigma\right)\bfp + \frac{1}{\beta}\bfB\wedge\bfE + \frac{1}{\rmPe}\rho\nabla\left[\frac{p}{\rho}\right]\right]\cdot\bfn
    \end{equation}
    In the ideal limit therefore, for $\rmRef, \rmRem, \rmPe  =  \infty$,
    \begin{equation}
        \frac{d\rmE}{dt}, \frac{d\rmH_{\rmM}}{dt}  =  0
    \end{equation}

    As detailed in the introduction, one can also interpret Gauss's law, (\ref{eqn:Gauss's law}), $\nabla\cdot\bfB  =  0$, discounted from the system at $t  >  0$ in the transient model, as a conserved property of the system, as it is enforced by Faraday's law, (\ref{eqn:Faraday's law}).
    