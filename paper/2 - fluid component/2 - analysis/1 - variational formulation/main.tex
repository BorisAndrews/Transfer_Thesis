\subsection{Variational Formulation}
    \BA{Introduction.}
    
    To apply the finite element method, one must first cast the model (\ref{eqn:current identity}–\ref{eqn:Gauss's law}) into a chosen variational form, by asserting that its inner product, in this case under the $L^{2}$ norm, is equal to $0$ upon integration by all function in some chosen function space, in this case ones derived from $\calD$, $\cdots$, $\calB$.

    \BA{What was the name for weak formulations where the test spaces don't equal the solution spaces? Hmmmmm. I should mention that though!}


    
    \renewcommand{\arraystretch}{1.5}
\addbibresource{references.bib}
\def\contra{
    \tikz[baseline, x=0.22em, y=0.22em, line width=0.032em]
    \draw (0,2.83)--(2.83,0) (0.71,3.54)--(3.54,0.71) (0,0.71)--(2.83,3.54) (0.71,0)--(3.54,2.83);
}

    \renewcommand{\arraystretch}{1.5}
\addbibresource{references.bib}
\def\contra{
    \tikz[baseline, x=0.22em, y=0.22em, line width=0.032em]
    \draw (0,2.83)--(2.83,0) (0.71,3.54)--(3.54,0.71) (0,0.71)--(2.83,3.54) (0.71,0)--(3.54,2.83);
}

    