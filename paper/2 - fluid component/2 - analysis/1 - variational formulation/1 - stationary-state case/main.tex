\subsubsection*{Stationary-State Case}
    Consider first the stationary-state system, as in \cite{LHF22}. The authors consider test functions for each equation in the space:\footnote{The notation therein differs slightly.}
    \begin{center}\begin{tabular}{ r l c | c }
        \multicolumn{2}{c}{Equation}  &  Index  &  Test space  \\
        \hline\hline
        Mass conservation  &  $0  =  \nabla\cdot\bfp$  &  (\ref{eqn:mass conservation})  &  $\bbP$  \\
        Momentum conservation  &  $\bfzero 
         =  - \nabla\cdot\left[\bfp^{\otimes 2}\right] - \nabla p + \frac{2}{\beta}\bfj\wedge\bfB + \frac{1}{\rmRef}\Delta\bfp$  &  (\ref{eqn:momentum conservation})  &  $\bbU$  \\
        \hline
        Current identity  &  $\bfzero  =  \frac{1}{\rmRem}\bfj - (\bfE + \bfp\wedge\bfB) + \rmRH\bfj\wedge\bfB$  &  (\ref{eqn:incompressible current identity})  &  $\bbJ$  \\
        \hline
        Ampère's law  &  $\bfzero  =  \nabla\wedge\bfB - \bfj$  &  (\ref{eqn:Ampère's law})  &  $\bbE$  \\
        Faraday's law  &  $\bfzero  =  \nabla\wedge\bfE$  &  (\ref{eqn:Faraday's law})  &  $\bbB$  \\
        Gauss's law  &  $\bfzero  =  \nabla\cdot\bfB$  &  (\ref{eqn:Gauss's law})  &  $\nabla\cdot\bbB$  \\
    \end{tabular}\end{center}
    This takes the variational formulation: \BA{(Clear up and clarify the notation a little here, even if this way is informative and convenient.)}
    \begin{align}
        \forall q \in \bbP,  0  &=  \langle\nabla\cdot\bfp, q\rangle_{\bfOmega}  \\
        \begin{split}
            \forall \bfq \in \bbU,  0  &=  \left.\left(\left\langle\bfp^{\otimes 2}, \nabla\bfq\right\rangle + \langle p, \nabla\cdot\bfq\rangle + \frac{2}{\beta}\langle\bfj\wedge\bfB, \bfq\rangle - \frac{1}{\rmRef}\langle\nabla\bfp, \nabla\bfq\rangle\right)\right|_{\bfOmega}  \\
            &\;\;\;\;\;\;\;\;\;\;\;\;\;\;\;\;\;\;\;\;\;\;\;\;\;\;\;\;\;\;\;\;  + \left.\left(- \langle(\bfp\cdot\bfn)\bfp, \bfq\rangle - \langle p, \bfq\cdot\bfn\rangle + \frac{1}{\rmRef}\langle\bfn\cdot\nabla\bfp, \bfq\rangle\right)\right|_{\bfGamma}
        \end{split}  \\
        \forall \bfk \in \bbJ,  0  &=  \left.\left(\frac{1}{\rmRem}\langle\bfj, \bfk\rangle - \langle\bfE, \bfk\rangle - \langle\bfp\wedge\bfB, \bfk\rangle + \rmRH\langle\bfj\wedge\bfB, \bfk\rangle\right)\right|_{\bfOmega}  \\
        \forall \bfF \in \bbE,  0  &=  \left.\left(\langle\bfB, \nabla\wedge\bfF\rangle - \langle\bfj, \bfF\rangle\tall\right)\right|_{\bfOmega} + \langle\bfB, \bfF\wedge\bfn\rangle_{\bfGamma}  \\
        \forall \bfC \in \bbB,  0  &=  \left.\left(\langle\nabla\wedge\bfE, \bfC\rangle + \langle\nabla\cdot\bfB, \nabla\cdot\bfC\rangle\tall\right)\right|_{\bfOmega}
    \end{align}
    4 equations in this model are linear: conservation of mass, \ref{eqn:mass conservation}, and the 3 EM laws, (\ref{eqn:Ampère's law}-\ref{eqn:Gauss's law}). Ensuring these properties hold \emph{strongly} (i.e. pointwise) is known to lead to more physically accurate and representative numerical results. \BA{[Ref, ...]} \BA{(Definitely want more than just this to talk about the motivation, etc. Would be nice to mention e.g. GEMPIC and over evidence that exact enforcement of these conditions is a good idea.)} As is done in \cite{LFM22}, this weak formulation allows one to ensure all but Ampère's law hold strongly provided certain subspace conditions hold on the function spaces, by considering certain test functions: \BA{(Where \emph{precisely} in \cite{LFM22}?)}
    \begin{center}\begin{tabular}{ c | c | c }
        Equation  &  Test function  &  Subspace condition  \\
        \hline\hline
        $0  =  \nabla\cdot\bfp$  &  $\nabla\cdot\bfp  \in  \bbP$  &  $\nabla\cdot\bbU  \leqslant  \bbP$  \\
        $0  =  \nabla\wedge\bfE$  &  $\nabla\wedge\bfE  \in  \bbB$  &  $\nabla\wedge\bbE  \leqslant  \bbB$  \\
        $0  =  \nabla\cdot\bfB$  &  $\bfB  \in  \bbB$  &
    \end{tabular}\end{center}
    Note the proof of pointwise $0  =  \nabla\cdot\bfB$ here relies on that of pointwise $0  =  \nabla\wedge\bfE$. Restricting to Hilbert spaces then, it is therefore natural to consider weak formulations where $\bbU$, $\bbP$ and $\bbE$, $\bbB$ form subcomplexes of $H\Lambda^{\bullet}(\bfOmega)$, with projection/interpolation maps $\pi_{*}$ such that the following diagrams commute:
    \begin{center}\begin{tikzpicture}[align = center, node distance = 4cm, auto]
        \node (Hcurl) at (0,   0)   {$\bfH(\bfcurl)$};
        \node (Hdiv1) at (3.5, 0)   {$\bfH(\rmdiv)$};
        \node (E)     at (0,   - 2) {$\bbE$};
        \node (B)     at (3.5, - 2) {$\bbB$};

        \draw[->] (Hcurl) -- (Hdiv1) node[above, midway] {$\bfcurl$};
        \draw[->] (E)     -- (B)     node[above, midway] {$\bfcurl$};
        \draw[->] (Hcurl) -- (E)     node[left,  midway] {$\pi_{\rmE}$};
        \draw[->] (Hdiv1) -- (B)     node[left,  midway] {$\pi_{\rmB}$};
        
        \node (Hdiv2) at (7,    0)   {$\bfH(\rmdiv)$};
        \node (L2)    at (10.5, 0)   {$L^{2}$};
        \node (U)     at (7,    - 2) {$\bbU$};
        \node (P)     at (10.5, - 2) {$\bbP$};

        \draw[->] (Hdiv2) -- (L2) node[above, midway] {$\rmdiv$};
        \draw[->] (U)     -- (P)  node[above, midway] {$\rmdiv$};
        \draw[->] (Hdiv2) -- (U)  node[left,  midway] {$\pi_{\rmU}$};
        \draw[->] (L2)    -- (P)  node[left,  midway] {$\pi_{\rmP}$};
    \end{tikzpicture}\end{center}

    In the \emph{compressible} case, the same test spaces for the corresponding equations are used, with the addition of the \emph{new} energy equation, which is naturally tested against the \emph{new} function space, $\bbD$:
    \begin{center}\begin{tabular}{ r l c | c }
        \multicolumn{2}{c}{Equation}  &  Index  &  Test space  \\
        \hline\hline
        $\vdots$  &  $\vdots$  &  $\vdots$  &  $\vdots$  \\
        Momentum conservation  &  $\bfzero 
         =  - \nabla\cdot\left[\frac{1}{\rho}\bfp^{\otimes 2}\right] - \nabla p + \frac{2}{\beta}\bfj\wedge\bfB + \cdots$  &  (\ref{eqn:momentum conservation})  &  $\bbU$  \\
        Energy conservation  &  $0  =  - \nabla\cdot\left[\frac{p}{\rho}\bfp\right] - p\nabla\cdot\left[\frac{1}{\rho}\bfp\right] + \cdots$  &  (\ref{eqn:energy conservation})  &  $\bbD$  \\
        \hline
        Current identity  &  $\bfzero  =  \frac{1}{\rmRem}\bfj - \left(\bfE + \frac{1}{\rho}\bfp\wedge\bfB\right) + \rmRH\bfj\wedge\bfB$  &  (\ref{eqn:current identity})  &  $\bbJ$  \\
        \hline
        $\vdots$  &  $\vdots$  &  $\vdots$  &  $\vdots$  \\
    \end{tabular}\end{center}
    giving the corresponding \emph{compressible} variational formulation equations:
    \begin{align}
        \begin{split}
            \forall \bfq \in \bbU,  0  &=  \left.\left(\left\langle\frac{1}{\rho}\bfp^{\otimes 2}, \nabla\bfq\right\rangle + \langle p, \nabla\cdot\bfq\rangle + \frac{2}{\beta}\langle\bfj\wedge\bfB, \bfq\rangle - \frac{1}{\rmRef}\langle\rho\bftau, \nabla_{s}\bfq)\rangle\right)\right|_{\bfOmega}  \\
            &\;\;\;\;\;\;\;\;\;\;\;\;\;\;\;\;\;\;\;\;\;\;\;\;  + \left.\left(- \left\langle\frac{1}{\rho}(\bfp\cdot\bfn)\bfp, \bfq\right\rangle - \langle p, \bfq\cdot\bfn\rangle + \frac{1}{\rmRef}\langle\rho\bftau\cdot\bfn, \bfq\rangle\right)\right|_{\bfGamma}
        \end{split}  \\
        \begin{split}
            \forall \sigma \in \bbD,  0  &=  \left(\left\langle\frac{p}{\rho}\bfp, \nabla\sigma\right\rangle - \left\langle p\nabla\cdot\left[\frac{1}{\rho}{\bfp}, \sigma\right]\right\rangle + \frac{1}{\rmRef}\left\langle\rho\bftau:\nabla\left[\frac{1}{\rho}\bfp\right], \sigma\right\rangle\right.  \\
            &\;\;\;\;\;\;\;\;\;\;\;\;\;\;\;\;\;\;\;\;\;\;\;\;\;\;\;\;\;\;\;\;\;\;\;\;\;\;\;\;\;\;\;\;\;\;\;\;\left.\left.+ \frac{2}{\beta\rmRem}\left\langle\|\bfj\|^{2}, \sigma\right\rangle - \frac{1}{\rmPe}\left\langle\rho\nabla\left[\frac{p}{\rho}\right], \nabla\sigma\right\rangle\right)\right|_{\bfOmega}  \\
            &\;\;\;\;\;\;\;\;\;\;\;\;\;\;\;\;\;\;\;\;\;\;\;\;  + \left.\left(- \left\langle\frac{p}{\rho}\bfp\cdot\bfn, \sigma\right\rangle + \frac{1}{\rmPe}\left\langle\rho\nabla\left[\frac{p}{\rho}\right]\cdot\bfn, \sigma\right\rangle\right)\right|_{\bfGamma}
        \end{split}  \\
        \forall \bfk \in \bbJ,  0  &=  \left.\left(\frac{1}{\rmRem}\langle\bfj, \bfk\rangle - \langle\bfE, \bfk\rangle - \left\langle\frac{1}{\rho}\bfp\wedge\bfB, \bfk\right\rangle + \rmRH\langle\bfj\wedge\bfB, \bfk\rangle\right)\right|_{\bfOmega}
    \end{align}
    Having left each of the linear systems unchanged, there is no reason to modify the subcomplex conditions from the incompressible case.
