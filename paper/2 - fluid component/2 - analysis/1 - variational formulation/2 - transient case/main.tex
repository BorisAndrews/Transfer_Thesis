\subsubsection*{Transient Case}
    \BA{Introduction.}

    Consider now the transient case, with the inclusion of the time derivatives, and discarding Gauss's law, (\ref{eqn:Gauss's law}).
    
    \BA{(Need some references for this paragraphs. Should trace the paper trail back from Giacomo's report.)} To derive transient schemes and timesteppers, the system can be interpreted through the lens of function/finite element spaces and weak formulations in both space \emph{and} time, i.e. on the domain $\bfOmega\otimes T^{k}$ over some timestep $T^{k}$, with the intention of constructing a timestepper from the resultant discretization. This is often done through the use of function/finite elements spaces that are tensor product spaces of one in space, $\bfOmega$, and time, $T^{k}$, giving timesteppers that often align with well-studied implicit Runge–Kutta methods. Approaching through the lens of finite elements in time facilitates the construction of structure-preserving discretizations directly from the weak formulation, using specially chosen test spaces. With the order in time of the system often being odd, these test spaces often differ from the solution spaces, such that the finite-element-in-time scheme is a Petrov–Galerkin method. For a further exposition and worked example of this technique in the case of dissipative schemes for the heat equation, see Appendix \ref{cha:constructing timesteppers}. A recent generic numerical implementation for timesteppers derived from finite elements in time, {\tt Fetsome}, building on the implicit Runge–Kutta method package {\tt Irksome} for the finite-element software {\tt Firedrake} is presented in the report \cite{La22}.

    Casting into weak form, consider test functions in the following test function spaces on $\bfOmega\otimes T^{k}$, similarly via the $L^{2}$ inner product, where the test function spaces $\bbF$, $\bbC$ are new spaces, distinct from $\bbP$, $\cdots$, $\bbB$:
    \begin{center}\begin{tabular}{ r l c | c }
        \multicolumn{2}{c}{Equation}  &  Index  &  Test space  \\
        \hline\hline
        Mass conservation  &  $- \partial_{t}\rho  =  \nabla\cdot\bfp$  &  (\ref{eqn:mass conservation})  &  $\bbP$  \\
        Momentum conservation  &  $\partial_{t}\bfp 
         =  - \bfp\cdot\nabla\bfu - \nabla p + \frac{2}{\beta}\bfj\wedge\bfB + \cdots$  &  (\ref{eqn:momentum conservation})  &  $\bbM$  \\
        Energy conservation  &  $\partial_{t}p  =  - \nabla\cdot[p\bfu] - p\nabla\cdot\bfu + \cdots$  &  (\ref{eqn:energy conservation})  &  $\bbD$  \\
        \hline
        Current identity  &  $\bfzero  =  \frac{1}{\rmRem}\bfj - \left(\bfE + \bfu\wedge\bfB\right) + \rmRH\bfj\wedge\bfB$  &  (\ref{eqn:current identity})  &  $\bbJ$  \\
        \hline
        Momentum identity  &  $\bfzero  =  \bfp - \rho\bfu$  &  (\ref{eqn:velocity identity})  &  $\bbU$  \\
        Temperature identity  &  $0  =  p - \rho\theta$  &  (\ref{eqn:temperature identity})  &  $\bbtheta$  \\
        \hline
        Ampère's law  &  $\bfzero  =  \nabla\wedge\bfB - \bfj$  &  (\ref{eqn:Ampère's law})  &  $\bbF$  \\
        Faraday's law  &  $\partial_{t}\bfB  =  - \nabla\wedge\bfE$  &  (\ref{eqn:Faraday's law})  &  $\bbC$  \\
    \end{tabular}\end{center}
    \BA{(What I might end up doing is doing an initial solve using different test spaces that don't necessarily conserve $\rmE$/$\rmH_{\rmM}$, then doing multigrid-in-time cycles using these conservational test spaces.)} This takes the variational formulation:
    
    \line

    \paragraph*{Compressible Transient Weak Formulation A}
    \begin{align}
        \forall q \in \bbP,  0  &=  \left.\left(\langle\partial_{t}\rho, q\rangle + \langle\nabla\cdot\bfp, q\rangle\tall\right)\right|_{\bfOmega\otimes T^{k}}  \\
        \begin{split}
            \forall \bfq \in \bbM,  0  &=  \left.\left(\langle\partial_{t}\bfp, \bfq\rangle - \langle\bfp\cdot\nabla\bfu, \bfq\rangle + \langle p, \nabla\cdot\bfq\rangle + \frac{2}{\beta}\langle\bfj\wedge\bfB, \bfq\rangle - \frac{1}{\rmRef}\langle\rho\bftau, \nabla_{s}\bfq)\rangle\right)\right|_{\bfOmega\otimes T^{k}}  \\
            &\;\;\;\;\;\;\;\;\;\;\;\;\;\;\;\;\;\;\;\;\;\;\;\;  + \left.\left(- \langle p, \bfq\cdot\bfn\rangle + \frac{1}{\rmRef}\langle\rho\bftau, \sym(\bfq\otimes\bfn)\rangle\right)\right|_{\bfGamma\otimes T^{k}}
        \end{split}  \\
        \begin{split}
            \forall \sigma \in \bbD,  0  &=  \left(\langle p, \partial_{t}\sigma\rangle + \langle p\bfu, \nabla\sigma\rangle - \langle p\nabla\cdot\bfu, \sigma\rangle + \frac{1}{\rmRef}\langle\rho\bftau:\nabla_{\rms}\bfu, \sigma\rangle\right.  \\
            &\;\;\;\;\;\;\;\;\;\;\;\;\;\;\;\;\;\;\;\;\;\;\;\;\;\;\;\;\;\;\;\;\;\;\;\;\;\;\;\;\;\;\;\;\;\;\;\;\left.\left.+ \frac{2}{\beta\rmRem}\left\langle\|\bfj\|^{2}, \sigma\right\rangle - \frac{1}{\rmPe}\langle\rho\nabla\theta, \nabla\sigma\rangle\right)\right|_{\bfOmega\otimes T^{k}}  \\
            &\;\;\;\;\;\;\;\;\;\;\;\;\;\;\;\;\;\;\;\;\;\;\;\;- \langle p, \sigma\rangle_{\bfOmega\otimes\partial T^{k}} + \left.\left(- \langle p\bfu\cdot\bfn, \sigma\rangle + \frac{1}{\rmPe}\langle\rho\nabla\theta\cdot\bfn, \sigma\rangle\right)\right|_{\bfGamma\otimes T^{k}}
        \end{split}  \\
        \forall \bfk \in \bbJ,  0  &=  \left.\left(\frac{1}{\rmRem}\langle\bfj, \bfk\rangle - \langle\bfE, \bfk\rangle - \langle\bfu\wedge\bfB, \bfk\rangle + \rmRH\langle\bfj\wedge\bfB, \bfk\rangle\right)\right|_{\bfOmega\otimes T^{k}}  \\
        \forall \bfv \in \bbU,  0  &=  \left.\left(\langle\bfp, \bfv\rangle - \langle\rho\bfu, \bfv\rangle\tall\right)\right|_{\bfOmega\otimes T^{k}}  \\
        \forall \eta \in \bbtheta,  0  &=  \left.\left(\langle p, \eta\rangle - \langle\rho\theta, \eta\rangle\tall\right)\right|_{\bfOmega\otimes T^{k}}  \\
        \forall \bfF \in \bbF,  0  &=  \left.\left(\langle\bfB, \nabla\wedge\bfF\rangle - \langle\bfj, \bfF\rangle\tall\right)\right|_{\bfOmega\otimes T^{k}} + \langle\bfB, \bfF\wedge\bfn\rangle_{\bfGamma\otimes T^{k}}  \\
        \forall \bfC \in \bbC,  0  &=  \left.\left(\langle\partial_{t}\bfB, \bfC\rangle - \langle\nabla\wedge\bfE, \bfC\rangle\tall\right)\right|_{\bfOmega\otimes T^{k}}
    \end{align}

    \line

    Similarly to the stationary-state case, one can seek to enforce mass conservation (\ref{eqn:mass conservation}) and Faraday's law (\ref{eqn:Faraday's law}) \emph{strongly} by considering certain test functions, and deriving conditions on the function spaces from there:
    \begin{center}\begin{tabular}{ c | c | c }
        Equation  &  Test function  &  Subspace condition  \\
        \hline\hline
        $0  =  \partial_{t}\rho + \nabla\cdot\bfp$  &  $\partial_{t}\rho + \nabla\cdot\bfp  \in  \bbP$  &  $\partial_{t}\bbD + \nabla\cdot\bbM  \leqslant  \bbP$  \\
        $0  =  \partial_{t}\bfB + \nabla\wedge\bfE$  &  $\partial_{t}\bfB + \nabla\wedge\bfE  \in  \bbC$  &  $\partial_{t}\bbB + \nabla\wedge\bbE  \leqslant  \bbC$
    \end{tabular}\end{center}
    Similarly restricting to Hilbert spaces, it is therefore natural to consider weak formulations where $(\bbM, \bbD)$, $\bbP$ and $(\bbE, \bbB)$, $(\bbC, *)$ form subcomplexes of $H\Lambda^{\bullet}\left(\bfOmega\otimes T^{k}\right)$, with projection/interpolation maps $\pi_{*}$ such that the following diagrams commute:
    \begin{center}\begin{tikzpicture}[align = center, node distance = 4cm, auto]
        \node (HL2)  at (0,   0)   {$\bfH\left(\begin{matrix} \bfcurl + \partial_{t} \\ * - \rmdiv \end{matrix}\right)$};
        \node (HL3a) at (5.5, 0)   {$\bfH(\rmdiv + \partial_{t})$};
        \node (EB)   at (0,   - 3) {$\bbE\oplus\bbB$};
        \node (*C)   at (5.5, - 3) {$\bbC\oplus*$};

        \draw[->] (HL2)  -- (HL3a) node[above, midway] {$\left(\begin{matrix} \bfcurl + \partial_{t} \\ * - \rmdiv \end{matrix}\right)$};
        \draw[->] (EB)   -- (*C)   node[above, midway] {$\left(\begin{matrix} \bfcurl + \partial_{t} \\ * - \rmdiv \end{matrix}\right)$};
        \draw[->] (HL2)  -- (EB)   node[left,  midway] {$\pi_{\rmE\rmB}$};
        \draw[->] (HL3a) -- (*C)   node[left,  midway] {$\pi_{\rmC*}$};
        
        \node (HL3b) at (9,   0)  {$\bfH(\rmdiv + \partial_{t})$};
        \node (HL4) at (12.5, 0)  {$L^{2}$};
        \node (MD)  at (9,   - 3) {$\bbM\oplus\bbD$};
        \node (P)  at (12.5, - 3) {$\bbP$};

        \draw[->] (HL3b) -- (HL4) node[above, midway] {$\rmdiv + \partial_{t}$};
        \draw[->] (MD)   -- (P)   node[above, midway] {$\rmdiv + \partial_{t}$};
        \draw[->] (HL3b) -- (MD)  node[left,  midway] {$\pi_{\rmM\rmD}$};
        \draw[->] (HL4)  -- (P)   node[left,  midway] {$\pi_{\rmP}$};
    \end{tikzpicture}\end{center}
    By Corollary \ref{cor:tensor product complex inclusion}, such subcomplexes can be constructed via a tensor product construction:
    \begin{center}\begin{tikzpicture}[align = center, node distance = 4cm, auto]
        \node (HL2) at (0,   0) {$\bfH\left(\begin{matrix} \bfcurl + \partial_{t} \\ * - \rmdiv \end{matrix}\right)$};
        \node (HL3) at (6.5, 0) {$\bfH(\rmdiv + \partial_{t})$};
        
        \node (Hcurl-L2) at (- 0.85, - 4.5) {$\bfH(\bfcurl)\otimes L^{2}$};
        \node (Hdiv-H1)  at (0.85,   - 3)   {$\bfH(\rmdiv)\otimes H^{1}$};
        \node (Hdiv-L2)  at (5.65,   - 4.5) {$\bfH(\rmdiv)\otimes L^{2}$};
        \node (L2-H1)    at (7.35,   - 3)   {$*$};

        \node (E) at (- 0.85, - 8.25) {$\bbE$};
        \node (B) at (0.85,   - 6.75) {$\bbB$};
        \node (C) at (5.65,   - 8.25) {$\bbC$};


        \node[ellipse, draw, dashed, minimum width = 5cm, minimum height = 3cm, rotate = 20] at (0,   - 3.75) {};
        \node[ellipse, draw, dashed, minimum width = 5cm, minimum height = 3cm, rotate = 20] at (6.5, - 3.75) {};


        \draw[->] (HL2) -- (HL3) node[above, midway] {$\left(\begin{matrix} \bfcurl + \partial_{t} \\ * - \rmdiv \end{matrix}\right)$};
        
        \draw[->] (HL2) -- (0,   - 2);
        \draw[->] (HL3) -- (6.5, - 2);
        
        \draw[->] (Hdiv-H1)  -- (Hdiv-L2) node[above, pos = 0.55] {$\partial_{t}$};
        \draw[->] (Hcurl-L2) -- (Hdiv-L2) node[above, pos = 0.65] {$\bfcurl$};
        
        \draw[->] (Hcurl-L2) -- (E) node[left, pos = 0.65] {$\pi_{\rmE}$};
        \draw[->] (Hdiv-H1)  -- (B) node[left, pos = 0.8]  {$\pi_{\rmB}$};
        \draw[->] (Hdiv-L2)  -- (C) node[left, pos = 0.65]  {$\pi_{\rmC}$};
        
        \draw[->] (B) -- (C) node[above, midway] {$\partial_{t}$};
        \draw[->] (E) -- (C) node[above, midway] {$\bfcurl$};
    \end{tikzpicture}\end{center}
    
    \begin{center}\begin{tikzpicture}[align = center, node distance = 4cm, auto]
        \node (HL3) at (0,   0) {$\bfH(\rmdiv + \partial_{t})$};
        \node (HL4) at (6.5, 0) {$L^{2}$};
        
        \node (Hdiv-L2) at (- 0.85, - 4.5) {$\bfH(\rmdiv)\otimes L^{2}$};
        \node (L2-H1)   at (0.85,   - 3)   {$L^{2}\otimes H^{1}$};
        \node (L2-L2)   at (5.65,   - 4.5) {$L^{2}\otimes L^{2}$};

        \node (M) at (- 0.85, - 8.25) {$\bbM$};
        \node (D) at (0.85,   - 6.75) {$\bbD$};
        \node (P) at (5.65,   - 8.25) {$\bbP$};


        \node[ellipse, draw, dashed, minimum width = 5cm, minimum height = 3cm, rotate = 20] at (0,   - 3.75) {};
        \node[ellipse, draw, dashed, minimum width = 5cm, minimum height = 3cm, rotate = 20] at (6.5, - 3.75) {};


        \draw[->] (HL3) -- (HL4) node[above, midway] {$\rmdiv + \partial_{t}$};
        
        \draw[->] (HL3) -- (0,   - 2);
        \draw[->] (HL4) -- (6.5, - 2);
        
        \draw[->] (L2-H1)   -- (L2-L2) node[above, pos = 0.55] {$\partial_{t}$};
        \draw[->] (Hdiv-L2) -- (L2-L2) node[above, pos = 0.65] {$\rmdiv$};
        
        \draw[->] (Hdiv-L2) -- (M) node[left, pos = 0.65] {$\pi_{\rmM}$};
        \draw[->] (L2-H1)   -- (D) node[left, pos = 0.8]  {$\pi_{\rmD}$};
        \draw[->] (L2-L2)   -- (P) node[left, pos = 0.65] {$\pi_{\rmP}$};
        
        \draw[->] (D) -- (P) node[above, midway] {$\partial_{t}$};
        \draw[->] (M) -- (P) node[above, midway] {$\rmdiv$};
    \end{tikzpicture}\end{center}
            
    \BA{TO ADD:
    \begin{itemize}
        \item  Proofs of energy/helicity/Gauss's law conservation (Modify weak formulations accordingly)
        \item  Notes about what conformity I need for my function spaces/how to handle the weak formulation for non-conforming discretizations
        \item  Continuity of helicity
        \item  Talk about 2.5D case
        \item  Upgrade tensor product complex commutative diagrams
        \item  Add further work environment for referencing to the further work chapter
        \item  Give exposition on constructing timesteppers from FE schemes in space and time (Appendix?)
        \item  Change horizontal lines to ```backslash'line'' objects
        \item  See if I can get a bound on the change in magnetic helicity, $\rmH_{\rmM}$, that looks something like $\int_{t^{k}}^{t^{k + 1}}\int_{\bfOmega}\|\bfj\|^{2} + \text{ ``something''}$
    \end{itemize}}
