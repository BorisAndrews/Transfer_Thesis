\subsection{Tokamak Plasmas: Physical Properties}
    Certain properties characterize the plasma in a tokamak:
    \begin{itemize}
        \item  {\bf Very high heat}: Plasma temperatures within a tokamak are on the order of $10^{8}\rmK$ \BA{[Ref]}; an order of magnitude \emph{higher} than that in the center of the sun, at around $1.5\times10^{7}\rmK$ \BA{[Ref]}.
        \item  {\bf Very strong EM fields}: The EM fields used to ionize tokamak plasmas have strengths on the order $1\rmT$ \BA{[Ref]}, with the world's most powerful magnets being those employed in the world's most powerful tokamaks \BA{[Ref]}.
        \item  {\bf Very low density}: \BA{(Why do tokamaks do this? I know there's these theoretical limits on particle density, but I've never seen anyone explain why we have to enforce this?)} Tokamak plasmas feature particle densities on the order of $10^{- 5}{\rm mol}^{- 1}$ \BA{(Check!)} \BA{(Is that even the right unit?)} \BA{[Ref]}. The quantity of hydrogen gas used during a JET pulse is often compared with the size \BA{(/mass? I'm not sure actually!)} of a postage stamp \BA{[Ref]}.
    \end{itemize}

    \BA{Some nice parameter scale estimates/values in that gyrokinetics manuscript- Multiscale Gyrokinetics for Rotating Tokamak Plasmas: Fluctuations, Transport and Energy Flows.}

    \BA{Dust particles from boundaries- leads onto next subsection.}

    